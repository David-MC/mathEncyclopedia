\documentclass[a4paper, 11pt]{article}

\usepackage[utf8]{inputenc}
\usepackage[T1]{fontenc}

\usepackage{amsmath, amsthm, amssymb}
\usepackage{hyperref}

\usepackage[retainorgcmds]{IEEEtrantools}

\newcommand{\tq}{:\ }
\newcommand{\parclprf}[1]{\begin{array}{r|}#1\\\hline\end{array}}

\begin{document}

\begin{enumerate}
	\item Determineu si els conjunts següents amb les operacions que s'indiquen són o no grups.
	\begin{enumerate}
		\item El conjunt dels nombres naturals $\mathcal{N}$ amb la suma.
		
			\begin{enumerate}
				\item (Operació interna) $\forall x,y \in \mathcal{N} (x + y \overset{?}{\in} \mathcal{N})$
				

				\item (Associativa) 
				\item (Element neutre) 
				\item (Inversos) 
			\end{enumerate}

		\item El conjunt dels nombres racionals $\mathcal{Q}$ amb:
		\begin{enumerate}
			\item La suma.
			\item El producte.
		\end{enumerate}
	\end{enumerate}

	\item[9] Sigui $G$ un grup cíclic d'ordre $n$, generat per un element $a$. Per a tot nombre enter $k$, determineu l'ordre del subgrup generat per $a^k$ i demostreu que $a^k$ és un generador de $G$ si, i només si, $mcd(k,n) = 1$.
	
		Sigui $|G| = n$ i $G = \langle a \rangle$, volem determinar $|\langle a^k \rangle| = ord (a^k)$:
		\[(a^k)^l = e \iff a^{kl} = e \implies n \mid kl\]
	
		Definim $d = mcd (n, k)$, $n = n' d$ i $k = k' d$, amb $mcd (n', k') = 1$, llavors:
		\[\left. \begin{array}{l}
			\displaystyle n \mid kl \implies n' \mid k'l \implies \frac{n}{mcd (k, n)} = n' \mid l \medskip \\
			(a^k)^{n'} = a^{kn'}= a^{k'n} = (a^n)^{k'} = e^{k'} = e
		\end{array} \right\} \implies ord (a^k) = \frac{n}{mcd (k, n)}\]

		Per tant, a partir de la conclusió anterior:
		\[G = \langle a^k \rangle \iff ord (a^k) = |G| = n \iff mcd(k, n) = 1\]

	\item[10] Sigui $G$ un grup cíclic d'ordre $n$.
	\begin{enumerate}
		\item Demostreu que tot subgrup de $G$ és cíclic.

			Sigui $H < G$ i $k \in \mathbb{Z}$ tal que $k = min (\{k \in \mathbb{Z} \mid a^k \in H\})$, volem veure $H = \langle a^k \rangle$: \medskip
			
			$\parclprf{\supset}$ Aquesta implicació és trivial ja que: 
			\[(a^k \in H) \land (H < G) \implies \langle a^k \rangle \subset H.\]

			$\parclprf{\subset}$ Sigui $a^m \in H$, com $m \in \mathbb{Z}$, podem descomposar $m = kq + r$ amb $(q, r \in Z) \land (0 \leq r < k)$:
			\[a^m = a^{kq+r} = (a^k)^q a^r\]
			Llavors, tenim $(a^k)^q \in \langle a^k \rangle \subset H$, per tant $a^m \in \langle a^k \rangle \iff a^r \in \langle a^k \rangle$.
			\[\left. \begin{array}{l}
				a^m = (a^k)^q a^r \implies a^r = a^m ((a^k)^q)^{-1} \in H \medskip \\
				k = min (\{k \in \mathbb{Z} \mid a^k \in H\}) \land (0 \leq r < k)
			\end{array} \right\} \implies r = 0\]
			
			Per tant, $a^r = e \in \langle a^k \rangle \implies a^m \in \langle a^k \rangle$.

		\item Demostreu que, per a cada divisor $d$ de $n$, existeix un únic subgrup de $G$ d'ordre $d$.
	
			Sigui $|G| = n$, $G = \langle a \rangle$ i $d \mid n$. Per l'exercici 9:
			\[\begin{aligned}
				|\langle a^k \rangle| = d & \iff ord (a^k) = d \\
				& \iff ord(a^k) = \frac{n}{mcd (k, n)} = d \\
				& \iff mcd(k, n) = \frac{n}{d}
			\end{aligned}\]
			
			Cal provar l'existencia d'un subgrup $A < G$ d'ordre $d$, i després que és l'únic.
			D'aquí deduïm $mcd (k, n) = \frac{n}{d} \implies \frac{n}{d} \mid k$, i llavors, $k =$
	\end{enumerate}
	
	\item[11] Sigui $\mu_n = \{z \in \mathbb{C} \tq z^n = 1\}$ el conjunt de les arrels $n$-èsimes de la unitat complexes. Demostreu que $\mu_n$ amb el producte de $\mathbb{C}$ és un grup cíclic.
	
	
	
	\item[12] Siguin $p$, $q$ nombres primers diferents i $r$, $s \geq 1$ nombres enters.
		\begin{enumerate}
			\item Determineu quants elements del grup $\mathbb{Z} / p \mathbb{Z} $ el generen.\medskip

				OPCIO 1
				
				

				OPCIO 2
				
				 Per l'exercici 9, sabem que 
				\[(|G| = n) \land (G = \langle a \rangle) \implies (G = \langle a^k \rangle \iff mcd(k, n) = 1)\]

				Llavors, per a tot $n$ tal que $|\mathbb{Z} / n \mathbb{Z}| = n$ i $\mathbb{Z} / n \mathbb{Z} = \langle 1 \rangle$:
				\[\mathbb{Z} / n \mathbb{Z} = \langle k \cdot 1 \rangle = \langle k \rangle \iff mcd (k, n) = 1\]

				Aplicant-ho, el cardinal del conjunt de generadors de $\mathbb{Z} / n \mathbb{Z}$ serà:
				\[\#\{x \in \{1, \dotsc , n\} \mid \langle x \rangle = \mathbb{Z} / n \mathbb{Z}\} = \#\{x \in \{1, \dotsc , n\} \mid mcd(x, n) = 1\}\]
				Aquest conjunt és equivalent al de la funció $\varphi$ d'Euler per a un $n$ qualsevol. Per tant, el cardinal del conjunt de generadors de $\mathbb{Z} / n \mathbb{Z}$ és $\varphi (n)$.

				CONCLUSIO

				Llavors, per al grup $\mathbb{Z} / p \mathbb{Z}$:
				\[\#\{x \in \{1, \dotsc , p\} \mid \langle x \rangle = \mathbb{Z} / p \mathbb{Z}\} = \varphi (p) = p-1\]

			\item Determineu quants elements del grup $\mathbb{Z} / p^r \mathbb{Z}$ el generen.\medskip

				Pel raonament de l'apartat anterior:
				\[\begin{aligned}
					\#\{x \in \{1, \dotsc , p^r\} \mid \langle x \rangle = \mathbb{Z} / p^r \mathbb{Z}\} & = \#\{x \in \{1, \dotsc , p^r\} \mid mcd(x, p^r) = 1\} \\
					& = \varphi (p^r) = p^{r-1}(p-1) = p^r-p^{r-1}
				\end{aligned}\]

			\item Determineu quants elements del grup $\mathbb{Z} / p^r q^s \mathbb{Z}$ el generen.
				
				Pel raonament de l'apartat anterior:
				\[\begin{aligned}
					\#\{x \in \{1, \dotsc , p^r q^s\} \mid \langle x \rangle = \mathbb{Z} / p^r q^s \mathbb{Z}\} & = \#\{x \in \{1, \dotsc , p^r q^s\} \mid mcd(x, p^r q^s) = 1\} \\
					& = \varphi (p^r q^s)
				\end{aligned}\]

				Llavors, apliquem la següent propietat de la funció $\varphi$ d'Euler:
				\[\forall m,n \in \mathbb{N} \tq (mcd(m,n) = 1 \implies \varphi (mn) = \varphi (m) \varphi (n))\]
				Per tant, com $mcd(p, q) = 1 \implies mcd(p^r,q^s) = 1$:
				\[\varphi (p^r, q^s) = \varphi (p^r) \varphi (q^s) = (p^{r-1} (p-1)) (q^{s-1}(q-1)) = (p^{r}-p^{r-1}) (q^s-q^{s-1}) \]
		\end{enumerate}
	
	\item[13] Siguin $\sigma$, $\tau \in S_9$ les permutacions següents:
		\[\sigma = \begin{pmatrix}
			1 & 2 & 3 & 4 & 5 & 6 & 7 & 8 & 9 \\
			2 & 9 & 1 & 8 & 7 & 6 & 3 & 4 & 5 \\
		\end{pmatrix},\qquad \tau = \begin{pmatrix}
			1 & 2 & 3 & 4 & 5 & 6 & 7 & 8 & 9 \\
			7 & 1 & 3 & 5 & 8 & 2 & 9 & 6 & 4 \\
		\end{pmatrix}\]

		\begin{enumerate}
			\item Calculeu $\sigma \tau$ i $\tau \sigma$. \medskip
			
				$\begin{aligned} \sigma \tau &= \begin{pmatrix}
						1 & 2 & 3 & 4 & 5 & 6 & 7 & 8 & 9 \\
						3 & 2 & 1 & 7 & 4 & 9 & 5 & 6 & 8 \\
					\end{pmatrix} \\
					\tau \sigma &= \begin{pmatrix}
						1 & 2 & 3 & 4 & 5 & 6 & 7 & 8 & 9 \\
						1 & 4 & 7 & 6 & 9 & 2 & 3 & 5 & 8 \\
				\end{pmatrix} \end{aligned}$\medskip

			\item Descomponeu $\sigma$ i $\tau$ com a producte de cicles disjunts, i també com a producte de transposicions; calculeu les seves signatures. \medskip
			
			$\displaystyle\begin{aligned}
				\sigma & = (1,2,9,5,7,3)(4,8) = (1,2)(2,9)(5,9)(5,7)(3,7)(4,8) \\
				\tau & = (1,7,9,4,5,8,6,2) = (1,7)(7,9)(4,9)(4,5)(5,8)(6,8)(2,6) \\
				\varepsilon (\sigma) & = 1 \\
				\varepsilon (\tau) & = -1
			\end{aligned}$\medskip

			\item Calculeu $\sigma^{2015}$.
				
				Siguin $\sigma_1 = (1,2,9,5,7,3)$ i $\sigma_2 = (4,8)$ cicles disjunts a $S_9$, d'ordre 6 i 2 respectivament, sabem que: 
				\[\sigma = \sigma_1 \sigma_2 = \sigma_2 \sigma_1,\qquad
				(\sigma_1)^6 = Id,\qquad
				(\sigma_2)^2 = Id\]
				
				Per tant, aplicant la descomposició de $\sigma$ en cicles disjunts i la commutativitat d'aquests cicles entre ells:
				
				$\displaystyle\begin{aligned}
					\sigma^n & = (\sigma_1 \sigma_2)^n = \overbrace{(\sigma_1 \sigma_2)(\sigma_1 \sigma_2) \:\cdots\:  (\sigma_1 \sigma_2)}^n \\ 
					& = \overbrace{(\sigma_1 \sigma_1 \:\cdots\: \sigma_1)}^n \overbrace{(\sigma_2 \sigma_2 \:\cdots\: \sigma_2)}^n = (\sigma_1)^n (\sigma_2)^n
				\end{aligned}$\medskip

				Llavors, a partit de les propietats anteriors:\medskip\par
				$\displaystyle\begin{aligned}
					\sigma^{2015} & = \sigma^{(6 \cdot 335 + 5)} = (\sigma^6)^{335} \sigma^5 \\
					& \qquad \sigma^6 = (\sigma_1 \sigma_2)^6 = (\sigma_1)^6 (\sigma_2)^6 = Id ((\sigma_2)^2)^3 = (Id)^3 = Id \\
					& = (Id)^{335} (\sigma)^5 = Id \ \sigma^5 = \sigma^5 = (\sigma_1 \sigma_2)^5 = (\sigma_1)^5 (\sigma_2)^5 \\
					& = (\sigma_1)^5 (\sigma_2)^{(2 \cdot 2 + 1)} = (\sigma_1)^5 ((\sigma_2)^2)^2 \sigma_2 = (\sigma_1)^5 (Id)^2 \sigma_2 \\
					& = (\sigma_1)^5 \sigma_2 = (1,3,7,5,9,2)(4,8)
				\end{aligned}$
		\end{enumerate}

	\item[14] Determineu la signatura de totes les permutacions de $S_3$. Determineu tots els subgrups de $S_3$.

		Sigui $t_1 = (1,2)$, $t_2 = (1,3)$, $t_3 = (2,3)$ , $c_1 = (1,2,3)$ i $c_2 = (1,3,2)$, llavors
		\[S_3 = \{Id,\ t_1,\ t_2,\ t_3,\ c_1,\ c_2\}\]
		\[\begin{aligned}
			\varepsilon (Id) & = 1 & \qquad \varepsilon (t_3) & = \varepsilon ((2,3)) = -1 \\
			\varepsilon (t_1) & = \varepsilon ((1,2)) = -1 & \qquad \varepsilon (c_1) & = \varepsilon ((1,2,3)) = \varepsilon ((1,2)(2,3)) = 1 \\
			\varepsilon (t_2) & = \varepsilon ((1,3)) = -1 & \qquad \varepsilon (c_2) & = \varepsilon ((1,3,2)) = \varepsilon ((1,3)(2,3)) = 1	
		\end{aligned}\]
		
		Per determinar tots els sugrups de $S_3$, pel teorema de Lagrange, ordre de qualsevol subgrup de $S_3$ ha de ser un divisor de l'ordre de $S_3$. Llavors, com $|S_3| = 6$ i 6 només té com a divisors 1, 2, 3 i 6, sabem que només els subconjunts d'ordre 1, 2, 3 i 6 poden ser subgrups. Sigui $A \subseteq S_3$:
		
		\begin{itemize}
			\item $|A| = 1$:
				\[\exists A = \{Id\}: ((Id\ Id = Id\ Id = Id) \implies A < S_3)\]
				A més, $\forall A \subseteq S_3: (A \neq \{Id\} \implies Id \not\in A \implies A \nless S_3)$, per tant, $\{Id\}$ és l'únic subgrup d'ordre 1.
			\item $|A| = 2$: Sabem que $Id \in A$ per a que $A$ sigui un subgrup de $S_3$, llavors podem distinguir dos casos:
				\begin{itemize}
					\item[$\circ$] $A = \{Id, t_i\}$, $i \in \{1, 2, 3\}$:
						\[\forall i \in \{1, 2, 3\}:\ ((Id\ t_i = t_i) \land (t_i\ t_i = Id) \implies A = \{Id, t_i\} < S_3)\]
					
					\item[$\circ$] $A = \{Id, c_i\}$, $i \in \{1, 2\}$:
						\[\forall i,j \in \{1,2\},\ i \neq j:\ ((c_i\ c_i = c_j) \land (c_j \not\in A) \implies A = \{Id, c_i\} \nless S_3)\]
				\end{itemize}
				Per tant, els subgrups d'ordre 2 són $\{Id, t_1\}$, $\{Id, t_2\}$ i $\{Id, t_3\}$.
			\item $|A| = 3$: Com al cas anterior, sabem que $Id \in A$, llavors tornem a distinguir tres casos:
				\begin{itemize}
					\item[$\circ$] $A = \{Id, t_i, t_j\}$, $i, j \in \{1, 2, 3\}$, $i \neq j$:
						\[\forall i, j \in \{1, 2, 3\},\ i \neq j \tq (\exists k \in \{1, 2\} \tq t_i\ t_j = c_k \not\in A) \implies A \nless S_3\]
					\item[$\circ$] $A = \{Id, c_1, c_2\}$:
						\[(c_1\ c_1 = c_2) \land (c_2\ c_2 = c_1) \land (c_1\ c_2 = c_2\ c_1 = Id) \implies A < S_3\]
					\item[$\circ$] $A = \{Id, c_i, t_j\}$, $i \in \{1, 2, 3\}$, $j \in \{1, 2\}$:
						\[\begin{aligned}
							\forall i &\in \{1, 2, 3\},\ j \in \{1, 2\} \tq \varepsilon (c_i)\ \varepsilon (t_j) = 1 \cdot (-1) = -1 \\
							&\implies \exists k \in \{1, 2, 3\} \tq ((c_i\ t_j = t_k) \land (c_i \neq Id \implies k \neq j)) \\
							&\implies A \nless S_3
						\end{aligned}\]
				\end{itemize}
				Per tant, l'únic subgrup d'ordre 3 és $\{Id, c_1, c_2\}$.
			\item $|A| = 6$: Sabem que l'únic subconjunt de $S_3$ amb el seu mateix ordre és ell mateix, llavors $A = S_3 < S_3$ és l'unic subgrup d'ordre 6.
		\end{itemize}
		Com a conclusió, els subgrups de $S_3$ són $\{Id\}$, $\{Id, t_1\}$, $\{Id, t_2\}$, $\{Id, t_3\}$, $\{Id, c_1, c_2\}$ i $S_3$.
	\item[15] Demostreu que, per a $n \geq 2$, $S_n$ té el mateix nombre de permutacions parelles que de permutacions senars.
		
		OPCIO1

		Definim $\tau \in S_n$ com una permutació senar qualsevol, amb $n \geq 2$, ja que a $S_1$ no hi ha permutacions senars. Llavors podem definir una aplicació entre les permutacions parelles i les senars de la manera següent:	
		\[\begin{aligned}
			f: A_n & \longrightarrow S_n \setminus A_n \\
			 \sigma & \longmapsto f(\sigma) = \tau \sigma = \gamma
		\end{aligned}\]
		
		Cal tenir present que:
		\[A_n = \{\sigma \in S_n \mid \varepsilon (\sigma) = 1\} \qquad S_n \setminus A_n = \{\sigma \in S_n \mid \varepsilon (\sigma) = -1\}\]
		
		Primer cal demostrar que l'aplicació està ben definida:
		\[\forall \sigma \in A_n \tq (\varepsilon (\tau\sigma) = \varepsilon (\tau) \cdot \varepsilon (\sigma) = (-1) \cdot 1 = -1 \implies \tau\sigma \in S_n \setminus A_n)\]
		
		Després, cal provar que és una aplicació bijectiva, i així haurem provat que $|A_n| = |S_n \setminus A_n|$. Per arribar a que $f$ és bijectiva, provarem que és injectiva i exhaustiva:
		\begin{itemize}
			\item $f$ és injectiva $\iff \forall \sigma, \sigma' \in A_n \tq (f(\sigma) = f(\sigma') \overset{?}{\implies} \sigma = \sigma')$:
				\[f(\sigma) = f(\sigma') \implies \tau \sigma = \tau \sigma' \implies \tau^{-1} \tau \sigma = \tau^{-1} \tau \sigma' \implies \sigma = \sigma'\]	
			\item $f$ és exhaustiva $\iff \forall \gamma \in S_n \setminus A_n \  \exists \sigma \in A_n \tq f(\sigma) \overset{?}{=} \gamma$:
				\[\begin{aligned}
					&\gamma \in S_n \setminus A_n \implies \tau^{-1} \gamma = \sigma \in S_n \\
					&(\varepsilon (\tau) = -1) \land (\varepsilon (Id) = 1) \land (\tau \tau^{-1} = Id) \implies \varepsilon (\tau^{-1}) = -1 \\
					&\varepsilon (\sigma) = \varepsilon (\tau^{-1} \gamma) = \varepsilon (\tau^{-1}) \varepsilon (\gamma) = 1 \implies \sigma \in A_n \\
					&\forall \gamma \in S_n \setminus A_n \tq (\tau^{-1} \gamma = \sigma \in A_n \implies \tau \tau^{-1} \gamma = \tau \sigma \implies \gamma = \tau \sigma = f(\sigma))
				\end{aligned}\]
		\end{itemize}

		Per tant, $f$ és una aplicació bijectiva, i $|A_n| = |S_n \setminus A_n|$.

		OPCIO 2
		
		Per a tot grup simètric $S_n$, podem definir:
		\[\begin{aligned}
			\varepsilon: S_n & \longrightarrow \{\pm 1\} \\
			\sigma & \longmapsto \varepsilon (\sigma)
		\end{aligned} \qquad \text{on } \: \varepsilon (\sigma) = \left\{\begin{array}{rl}
			1 & \text{si } \sigma \in A_n \\
			-1 & \text{si } \sigma \in S_n \setminus A_n
		\end{array}\right.\]

		Com ja hem demostrat a teoria, aquesta aplicació és un morfisme de grups i és exhaustiva per tot $n \geq 2$. A més:
		\[\left.\begin{array}{l}
			\forall \sigma \in A_n \tq \varepsilon (\sigma) = 1 \\
			\forall \tau \in S_n \setminus A_n \tq \varepsilon (\tau) = -1\\
			\text{L'element neutre de } \{\pm 1\} \text{ és } 1
		\end{array}\right\} \implies Ker(\varepsilon) = A_n\]
		
		Llavors, teorema d'isomorfia, tenim que:
		\[\begin{aligned}
			\tilde{\varepsilon}: S_n \big/ A_n & \longrightarrow \{\pm 1\} \\
			[\bar{\sigma}] & \longmapsto \varepsilon (x)
		\end{aligned}\]

		També per propietats demostrades a teoria, totes les classes d'una relació d'equivalencia associada a un subgrup tenen el mateix cardinal que el subgrup. Com, $\tilde{\varepsilon}$ només pot ser 1 o -1, hi ha dos classes associades al subgrup $A_n$, i són el mateix $A_n$ i $S_n \setminus A_n$. Llavors aquestes dos classes han de tenir el mateix cardinal; per tant $|A_n| = |S_n \setminus A_n|$.

		OPCIO 3
		
		Tot grup simètric $S_n$, amb $n \geq 2$, conté almenys una permutació parella, per exemple $Id$, i almenys una senar, per exemple el cicle $(1, 2)$. A més, com $|S_n| = n!$, hi haura un nombre finit de permutacions senar i parelles.
		
		Siguin $A_n = \{\sigma_1, \sigma_2, \dotsc , \sigma_r\}$ el conjunt de totes les permutacions parelles de $S_n$, i $S_n \setminus A_n = I = \{\tau_1, \tau_2, \dotsc , \tau_s\}$ el conjunt de totes les permutacions senars de $S_n$, cal veure $r = s$:
		\[\left.\begin{aligned}
			\forall i \in \{1, \dotsc, r\} \tq \varepsilon (\sigma_i\tau_1) = -1 \implies \sigma_i\tau \in I & \implies r \leq s \\
			\forall j \in \{1, \dotsc, s\} \tq \varepsilon (\tau_j\tau_1) = 1 \implies \tau_j\tau_1 \in A_n & \implies r \geq s
		\end{aligned}\right\} \implies r = s\]

		Per tant, com $r = s$, $|A_n| = |I| = |S_n \setminus A_n|$.

	\item[17] Demostreu que $S_n$ admet el sistema de generadors següents:
		\begin{enumerate}
			\item $A = \{(1,2),\ (1,3), \dotsc ,\ (1,n)\}$\medskip
			
				Volem demostrar $S_n = \langle A\rangle$, sigui $A = \{(1,a) \in S_n\}$. $\langle A\rangle \subseteq S_n$ és trivial, ja que $\forall \sigma \in A \tq \sigma \in S_n$. 
				
				Només ens cal demostrar que $S_n \subseteq \langle A \rangle$, per fer-ho, utilitzarem el fet que $\forall a,b \in \mathbb{N} \tq (a \neq b) \land (S_n \subseteq \langle (a,b) \rangle)$, ja que, qualsevol permutació és pot expressar com a producte de transposicions.

				Llavors, només cal provar que $\langle \{(a,b) \in S_n\} \rangle \subseteq \langle A \rangle$:
				\[\exists (1,a), (1,b) \in A \tq (a,b) = (1,a)(1,b)(1,a) \implies (a,b) \in \langle A \rangle\]

			\item $B = \{(1,2),\ (2,3), \dotsc ,\ (n-1,n)\}$\medskip
				
				Volem demostrar $S_n = \langle B \rangle$, sigui $B = \{(a,a+1) \in S_n\}_a$. $\langle B \rangle \subseteq S_n$ és trivial, ja que $\forall \sigma \in B \tq \sigma \in S_n$.
				
				Només ens cal demostrar que $S_n \subseteq \langle B \rangle$, per fer-ho:

				OPCIO 1
				\[S_n \subseteq \langle B \rangle \iff S_n = \langle \{(1,a) \in S_n\}_a \rangle = \langle A \rangle \subseteq \langle B \rangle\] 
				Llavors, només cal provar que $\langle A \rangle \subseteq \langle B \rangle$:
				\[\begin{aligned}
					(1,a) & \overset{?}{=} [(1,2)(2,3) \cdots (a-2,a-1)] (a-1, a) [(a-1,a-2) \cdots (3,2)(2,1)] \\
					& = (1, 2, 3, \dotsc , a-2, a-1) (a-1, a) (a-1, a-2, \dotsc , 3, 2, 1) \\
					& = \sigma (a-1, a) \sigma^{-1} \\
					& = (\sigma (a-1), \sigma (a)) \\
					& = (1,a)
				\end{aligned}\]

				OPCIO 2
				\[S_n \subseteq \langle B \rangle \iff S_n = \langle \{(a,b) \in S_n\}_{a, b} \rangle \subseteq \langle B \rangle\]

				Llavors, només cal provar que $\langle \{(a,b) \in S_n\}_{a, b} \rangle \subseteq \langle B \rangle$:
				\begin{IEEEeqnarray*}{rCl}
					(a, b) & \overset{?}{=} & [(a, a+1)(a+1, a+2) \cdots (b-2,b-1)] (b-1, b) \\ 
					&& \quad [(b-1,b-2) \cdots (a+2, a+1)(a+1, a)] \\
					& = & (a, a+1, a+2, \dotsc , b-2, b-1) (b-1, b) \\
					&& \quad (b-1, b-2, \dotsc ,a+2, a+1, a) \\
					& = & \tau (b-1, b) \tau^{-1} \\
					& = & (\tau (b-1), \tau (b)) \\
					& = & (a, b)
				\end{IEEEeqnarray*}

			\item $C = \{(1,2, \dotsc ,n),\ (1,2)\}$\medskip

				Volem demostrar $S_n = \langle C \rangle$. $\langle C \rangle \subseteq S_n$ és trivial, ja que $\forall \sigma \in C \tq \sigma \in S_n$.

				Només ens cal demostrar que $S_n \subseteq \langle C \rangle$, per fer-ho:

				OPCIO 1
				\[S_n \subseteq \langle C \rangle \iff S_n = \langle \{(a,a+1) \in S_n\} \rangle = \langle B \rangle \subseteq \langle C \rangle\] 
				
				Llavors, només cal provar que $\langle B \rangle \subseteq \langle C \rangle$:
				\begin{IEEEeqnarray*}{rCl}
					(a, a+1) & \overset{?}{=} & (1, 2, \dotsc , n)^{a-1} (1, 2) (1, 2, \dotsc , n)^{n-a+1} \\
					& = & (1, 2, \dotsc , n)^{a-1} (1, 2) (1, 2, \dotsc , n)^{-(a-1)} \\
					& = & \gamma^{a-1} (1,2) (\gamma^{a-1})^{-1} \\
					& = & (\gamma^{a-1} (1), \gamma^{a-1} (2)) \\
					& = & (a, a+1)
				\end{IEEEeqnarray*}

				OPCIO 2
				\[S_n \subseteq \langle C \rangle \iff S_n = \langle \{(1,a) \in S_n\} \rangle = \langle A \rangle \subseteq \langle C \rangle\] 

				Llavors, només cal provar que $\langle A \rangle \subseteq \langle C \rangle$:
				\begin{IEEEeqnarray*}{rCl}
					(1, a) & \overset{?}{=} & ((1, 2)(1, 2, \dotsc , n))^{a-2}(1, 2)((1,2)(1, 2, \dotsc , n))^{n-a+2} \\
					& = & (2, 3, \dotsc , n)^{a-2} (1, 2) (2, 3, \dotsc , n)^{-(a-2)} \\
					& = & \sigma^{a-2} (1, 2) \sigma^{-(a-2)} \\
					& = & (\sigma^{a-2} (1), \sigma^{a-2} (2)) \\
					& = & (1, a)
				\end{IEEEeqnarray*}

				OPCIO 3
				\[S_n \subseteq \langle C \rangle \iff S_n = \langle \{(a,b) \in S_n\} \rangle \subseteq \langle C \rangle\] 
				
				Llavors, només cal provar que $\langle \{(a,b) \in S_n\} \rangle \subseteq \langle C \rangle$:
				\begin{IEEEeqnarray*}{rCl}
					(a, b) & \overset{?}{=} & (1, 2, \dotsc , n)^{a-1} (2, \dotsc , n)^{b-a-1} (1, 2) (2, \dotsc , n)^{-(b-a-1)} (1, 2, \dotsc , n)^{-(a-1)} \\
					& = & \tau^{a-1} \sigma^{b-a-1} (1,2) \sigma^{-(b-a-1)} \tau^{-(a-1)} \\
					& = & \tau^{a-1} (\sigma^{b-a-1} (1), \sigma^{b-a-1} (2)) \tau^{-(a-1)} \\
					& = & \tau^{a-1} (1, b-a+1) \tau^{-(a-1)} \\
					& = & (\tau^{a-1} (1), \tau^{a-1} (b-a+1)) \\
					& = & (a, b)
				\end{IEEEeqnarray*}

		\end{enumerate}

	\item[22] Demostreu que, si $G$ és un grup, el seu centre $Z(G) := \{g \in G \tq gh = hg, \text{ per a tot } h \in G\}$ és un subgrup normal de $G$.
		
		Primer, comprovem que $Z(G)$ és un subgrup de $G$.
		\[\begin{array}{l}
			\forall g \in Z(G),\ h \in G \tq ((gh = hg \implies hg^{-1} = g^{-1}h) \implies g^{-1} \in Z(g)) \\
			\forall x, y \in Z(G),\ h \in G \tq (xy^{-1}h = xhy^{-1} = hxy^{-1} \implies xy^{-1} \in Z(G))
		\end{array}\]

		Per tant, $Z(G)$ és subgrup de $G$, i només cal comprovar que sigui normal:
		\[\begin{aligned}
			Z(G) \text{ és subgrup normal} & \iff \forall h \in G \tq h Z(G) \overset{?}{=} Z(G) h \\
			& \iff \forall h \in G \tq (h Z(G) \overset{?}{\subset} Z(G) h) \land (h Z(G) \overset{?}{\supset} Z(G) h)
		\end{aligned}\]
		$\begin{array}{lr@{\ }l}
			\parclprf{\subset} \quad & x \in h Z(G) & \implies \exists z \in Z(G) \tq x = hz \\
			&& \implies x = zh \in Z(G) h \medskip \\
			\parclprf{\supset} \quad & x \in Z(G) h & \implies \exists z \in Z(G) \tq x = zh \\
			&& \implies x = hz \in h Z(G) 
		\end{array}$

	\item[24] Demostreu que, si $n \geq 3$, el centre de $S_n$ només conté la identitat.

		OPCIO 1

		Suposem $\exists \sigma \in Z(S_n) \tq ((\sigma \neq Id) \land (n \geq 3) \implies \forall \tau \in S_n \tq \tau\sigma = \sigma\tau)$:
		\[\begin{aligned}
			n \geq 3 \implies & \exists a, b, c \in \mathbb{N}_n \tq (a \neq b) \land (a \neq c) \land (b \neq c) \land (\sigma (a) = b) \\ 
			\implies & \exists \tau \in S_n \tq (\tau (b) = c) \land (\tau (a) = a) \\
			\implies & (\tau\sigma)(a) = \tau (\sigma (a)) = \tau (b) = c \\
			& (\sigma\tau)(a) = \sigma (\tau (a)) = \sigma (a) = b \\
			\implies & \tau\sigma \neq \sigma\tau \implies \bot \\ 
			\implies & \nexists \sigma \in Z(S_n) \tq (\sigma \neq Id) \land (n \geq 3)
		\end{aligned}\]
	
		OPCIO 2

		Suposem $\exists \sigma \in S_n \tq \sigma \neq Id$, i volem veure que $\exists \tau \in S_n \tq \sigma \tau = \tau \sigma$:
		\[\sigma \neq Id \implies \exists \alpha, \beta \in S_n, (\alpha \neq \beta \land \alpha \sigma = \beta)\]
		
		Llavors, definim $\tau, \gamma \in S_n$ amb $\gamma \neq \alpha$, $\gamma \neq \beta$, com:
		\[\left\{\begin{aligned}
			\alpha \tau & = \alpha \\
			\beta \tau & = \gamma
		\end{aligned}\right.\]
		Aleshores:
		\[\left.\begin{aligned}
			\alpha (\sigma \tau) = (\alpha \sigma) \tau = \beta \tau = \gamma \\
			\alpha (\tau \sigma) = (\alpha \tau) \sigma = \alpha \sigma = \beta
		\end{aligned}\right\} \implies \alpha (\sigma \tau) \neq \alpha (\tau \sigma) \implies \sigma \tau \neq \tau \sigma\]

		OPCIO 3

		Suposem $\exists \sigma \in S_n \tq (\sigma \neq Id) \land (\forall \tau \in S_n \tq \sigma \tau = \tau \sigma)$. Per demostrar que $\sigma \not\in Z (S_n)$ només cal demostrar que $\exists \sigma' \in S_n$ tal que:
		\[\begin{aligned}
			\sigma \tau = \tau \sigma & \implies \sigma = \tau \sigma \tau^{-1} \\
			\sigma' = \tau \sigma \tau^{-1} \neq \sigma & \implies \sigma \not\in Z (S_n)
		\end{aligned}\]
		Per l'exercici 16, sabem que  

		
\end{enumerate}
\end{document}
