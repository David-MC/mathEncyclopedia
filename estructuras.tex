\documentclass[a4paper, 11pt]{article}



\usepackage{amsmath, amsthm, amssymb}
\usepackage{fitch}

\usepackage{hyperref}

\theoremstyle{plain}
\newtheorem{prop}{Proposition}[section]


\theoremstyle{definition}
\newtheorem{defn}[prop]{Definition}

\theoremstyle{remark}

\newenvironment{flusheqs}
 {\noindent$\begin{array}{@{}>{\displaystyle}l@{}}}
 {\end{array}$\par}

\newenvironment{fproof}[1][~]
	{\begin{proof} #1\\%
	\belowdisplayskip=-15pt%
	\noindent$\begin{array}{@{}>{\displaystyle}l@{}}%
	\begin{fitch}}
	{\end{fitch}\end{array}$%
	\par\vspace{-13pt}%
	\end{proof}}
	%\ignorespaces}
	%\ignorespacesafterend}



\begin{document}

\section{Groups}

\begin{defn}
	Let $G = (G, e)$ be a monoid,\\ $G$ is a group$ \iff 
	\forall a \in G:\ \exists a^{-1} \in G:\ a^{-1}a = aa^{-1} = e$
\end{defn}

\begin{prop}
	Let $G$ be a group, $\forall c \in G:\ (cc = c \implies c = e)$
\end{prop}

\begin{fproof}
	\fh \forall c \in G:\ cc = c \\
	\fa \implies c^{-1}cc = c^{-1}c \\
	\fa \implies (c^{-1}c)c = c^{-1}c \\
	\fa \implies ec = e \\
	\fa \implies c = e \\
	\forall c \in G:\ (cc = c \implies c = e)
\end{fproof}
\begin{fproof}[$\forall c \in G:$]
	\fh cc = c \\
	\fa c^{-1}(cc) = c^{-1}c \\
	\fa (c^{-1}c)c = c^{-1}c \\
	\fa ec = e \\
	\fa c = e \\
	(cc = c \implies c = e)
\end{fproof}
\begin{prop}
	Let $G$ be a group,
\end{prop}

\end{document}
