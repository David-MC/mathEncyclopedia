\chapter{Geometria lineal}

\section{$\mathsection$Espais afins}
\label{ea}

\begin{defn}[Espai afí]
	Sigui $K$ un cos, $E$ un $K$-espai vectorial de dimensió finita i $A$ un cojunt tal que $A\neq\phi$. Un espai afí sobre un cos $K$ de espai director $E$ és una terna $(A,E,\phi)$ tal que:
	$$\begin{array}{rcl}
		\phi :A\times E	& \longrightarrow	& A 				\\
		(p,\vv{v})	& \longmapsto		& \phi (p,\vv{v}):=p+\vv{v}	\\
	\end{array}$$
	Anomenarem punts als elements del conjunt $A$. A més, es verifiquen els següents axiomes:
	\begin{itemize}
		\item[(A1)] $\forall p\in A$, la aplicació
		$$\begin{array}{rcl}
			\phi_{p} :E	& \longrightarrow	& A 		\\
			\vv{v}		& \longmapsto		& p+\vv{v}	\\
		\end{array}$$
		és bijectiva.
		\item[(A2)] $\forall p\in A,\: \vv{u},\vv{v}\in E$,
		\begin{center}
			$\phi (p,\vv{u}+\vv{v})=\phi (\phi (p,\vv{u}),\vv{v})$ \\
			$p+(\vv{u}+\vv{v})=(p+\vv{u})+\vv{v}$
		\end{center}
	\end{itemize}
	Sigui $dim(E)=n$, definirem la dimensió de l'espai afí $A$ com $dim(A)=dim(E)=n$.
\end{defn}

\begin{exmp}
	$dim(A)=0\Leftrightarrow E=\{ 0\}\Leftrightarrow A=\{ p\}$
\end{exmp}
\begin{exmp}
	$dim(A)=1\Leftrightarrow E\cong K\Rightarrow A$ és una recta	
\end{exmp}

\begin{lem}\begin{enumerate}
	\item $\forall p,q\in A,\: \exists !\: \vv{v}\in E$ tal que $p=q+\vv{v}$, i definim aquest vector com $\vv{v}=\vv{qp}$
	\item (Llei de Chasles) $\vv{pr}=\vv{pq}+\vv{qr}$
\end{enumerate}\end{lem}
\begin{proof}\begin{enumerate}
	\item Per $(A1)$ sabem que, \\
	$\begin{array}{rcl}
		\phi_{p} :E	& \longrightarrow	& A 		\\
		\vv{v}		& \longmapsto		& q+\vv{v}	\\
	\end{array}$ és bijectiva $\Rightarrow \exists ! \vv{v}$ tal que $\vv{v}=\phi_{p}^{-1}(p)=\phi_{p}^{-1}(q+\vv{v})$
	\item 
\end{enumerate}\end{proof}

\subsection{Varietats lineals}

\begin{defn}
	Sigui $A$ un espai afí, amb $dim(A)=n$, i de espai director E. Una (sub)varietat lineal de $A$ és un subconjunt $\mathbb{L}\subseteq A$ de la forma $\mathbb{L}=p+F$, amb $p\in A$ com a punt de pas de $\mathbb{L}$ i $F$ un subespai vectorial de $E$ com a espai director
	$$\mathbb{L}=\{ q\in A |\exists \vv{v}\in F, q = p + v\} \subseteq A$$
	Si una subvarietat lineal és de dimensió $n-1$, l'anomenem hiperpla.
\end{defn}

\begin{lem}
	Sigui $A$ un espai afí d'espai director $E$ i $\mathbb{L}$ una subvarietat lineal de $A$ amb espai director $F$, llavors $dim(\mathbb{L})=dim(F)\leq dim(E)=dim(A)$.
\end{lem}

\begin{prop}\begin{enumerate}
	\item Si $q\in \mathbb{L}$, llavors $\mathbb{L}=p+F=q+F$, i per tant el punt de pas d'una subvarietat lineal no és únic.
	\item Sigui $F=\{ \vv{pq}\in E$ on $p,q\in \mathbb{L}\}\Rightarrow$ El espai director d'una subvarietat lineal és únic.
\end{enumerate}\end{prop}
\begin{proof}\begin{enumerate}
	\item $q\in \mathbb{L}\Rightarrow\exists\vv{v}\in F$ tal que $q=p+\vv{v}\Rightarrow p=q-\vv{v}$ \\
	$p+F\subseteq q+F$? $w\in F$, $p+w=q-v+w=q+(-v+w)\in q+F$ \\
	$q+F\subseteq p+F$? $w\in F$, $q+w=(p+v)+w=p+(v+w)\in p+F$
	\item
\end{enumerate}\end{proof}

\begin{prop}
	Sigui $f:E\longrightarrow F$ una aplicació lineal i $b\in Im(f)=f(E)\subseteq F$, llavors $f^{-1}(b)=a+Nuc(f)$ i $f(a)=b$ i $f(a')=b$, amb $a\in E$, tal que $f^{-1}(b)$ és una varietat lineal de E amb $dim(Nuc(f))=dim(E)-rang(f)$.
\end{prop}
\begin{exmp}\begin{enumerate}
	\item El sistema d'equacions \\
	$$\left.\begin{array}{rcl}
		x-3y+2z	& =	& 	1	\\
		y-z	& =	&	0	\\
	\end{array}\right\}$$
	defineix una subvarietat lineal de $\mathbb{R}^{3}$ \\
	$\mathbb{L}=(2,1,1)+<(1,1,1)>$ \\
	$f:\mathbb{R}^{3}\longrightarrow \mathbb{R}^{2}\hspace{5em}b=(1,0)$
	\item Una sola equació
	$$\begin{array}{rcl}
		x-3y+4z+w	& =	& 6	\\
	\end{array}$$
	$f^{-1}(6)=\mathbb{L}=(6,0,0,0)+Nuc(f)=(6,0,0,0)+<(...),(...),(...)>$ \\
	$f:\mathbb{R}^{3}\longrightarrow \mathbb{R}\hspace{5em}b=6$
\end{enumerate}\end{exmp}

\subsection{Referències cartesianes}
\label{ss_refcar}

\begin{defn}
	Un sistema de coordenades de $A$ ve donat per:
	\begin{itemize}
		\item Un punt $P$ de $A$, que anomenem l'origen del sistema.
		\item Una base $\vv{e_{1}},\vv{e_{2}},...,\vv{e_{n}}$ de $E$.
	\end{itemize}
	$$\mathcal{R}=\{P;\vv{e_{1}},\vv{e_{2}},...,\vv{e_{n}}\}$$
	
	Com la aplicació \\
	$$\begin{array}{rcl}
		\phi_{p} :E	& \longrightarrow	& A 		\\
		\vv{u}		& \longmapsto		& \phi_{p}(\vv{u})=p+\vv{u}	\\
	\end{array}$$
	és bijectiva. \\
	$\forall q\in A, \exists !\vv{u}\in E$ tal que $\phi_{p}=p+\vv{u}=q$, com $\vv{e_{1}},\vv{e_{2}},...,\vv{e_{n}}$ és base de $E$
	$$\vv{u}=\lambda_{1}\vv{e_1}+...+\lambda_{n}\vv{e_{n}}$$
	i així
	$$q=p+\lambda_{1}\vv{e_{1}}+...+\lambda_{1}\vv{e_{n}}$$ 
	Anomenem les coordenades del punt $q$ en la referència $\mathcal{R}$ a $(\lambda_{1},\lambda_{2},...,\lambda_{n})$. Per tant les coordenades de q són els components del vector $\vv{pq}$ en la base $\vv{e_{1}},\vv{e_{2}},...,\vv{e_{n}}$ de $E$.
\end{defn}

\subsection{Equacions paramètriques de una varietat lineal}
\label{ss_eqimvl}

\begin{defn}
	Sigui $A$ un espai afí de espai director $E$, $\mathcal{R}=\{ P;\vv{e_{1}},...,\vv{e_{n}}\}$ una referència, $\mathbb{L}=a+F$ una subvarietat lineal de $A$ i el conjunt de vectors $\{\vv{v_{1}},...,\vv{v_{r}}\}$ una base de $F$. En la referencia $\mathcal{R}$ el punt de pas de $\mathbb{L}$ s'expressarà:
	$$a=P+\sum_{i=1}^{n}a_{i}\vv{e_{i}}$$
	Per tant les coordenades de $a$ en la referència $\mathcal{R}$ seràn $(a_{1},...,a_{n})$. Ara expressem la base de $F$ en la de $E$ que hem utilitzat a la referència:
	$$\vv{v_{j}}=\sum_{i=1}^{n}v_{j}^{i}\vv{e_{i}},\: j=1,...,r$$
	Llavors, sigui $\vv{w}=\sum_{i=1}^{r}w^{i}\vv{v_{i}}\in F$, podem expressar un punt qualsevol $x\in\mathbb{L}$ com:
	$$x=a+\vv{w}=P+\sum_{i=1}^{n}a_{i}\vv{e_{i}}+\sum_{j=0}^{r}w^{j}\vv{v_{j}}=P+\sum_{i=1}^{n}a_{i}\vv{e_{i}}+\sum_{j=0}^{r}w^{j}(\sum_{k=0}^{n}v_{j}^{k}\vv{e_{k}})=P+\sum_{i=0}^{n}(a_{i}+\sum_{j=0}^{r}w^{j}v_{j}^{i})\vv{e_{i}}$$
	Per tant, si anomenem $(x_{1},...,x_{n})$ a les coordenades de $x$ en la referència $\mathcal{R}$, tenim:
	$$\begin{array}{rcl}
		x_{1}	& =		& a_{1}+w^{1}v_{1}^{1}+w^{2}v_{2}^{1}+...+w^{r}v_{r}^{1}	\\
		x_{2}	& =		& a_{1}+w^{1}v_{1}^{2}+w^{2}v_{2}^{2}+...+w^{r}v_{r}^{2}	\\
			& \vdots	&								\\
		x_{n}	& =		& a_{n}+w^{1}v_{1}^{n}+w^{2}v_{2}^{n}+...+w^{r}v_{r}^{n}	\\
	\end{array}$$
	Les equacions formades per les coordenades de $x$ a partir de les coordenades de $a$, $w$ i la base de $F$ en la referència $\mathcal{R}$ les anomenem equacions paramètriques de $\mathbb{L}$.
\end{defn}
\begin{exmp}
	Sigui $\mathbb{L}=a+F$ una recta en $\mathbb{R}^{5}$, en la referència $\mathcal{R}$, $a=(1,3,5,2,8)$, $F=<(3,2,0,6,9)>$, les coordenades d'un punt $x=(x_{1},x_{2},x_{3},x_{4},x_{5})\in\mathbb{L}$ vindràn expressades per les equacions:
	$$\left.\begin{array}{rcl}
		x_{1}	& =	& 1+3\vv{w}	\\
		x_{2}	& =	& 3+2\vv{w}	\\
		x_{3}	& =	& 5+0\vv{w}	\\
		x_{4}	& =	& 2+6\vv{w}	\\
		x_{5}	& =	& 8+9\vv{w}	\\
	\end{array}\right\}\Longleftrightarrow \mathbb{L}=(1,3,5,2,8)+<(3,2,0,6,9)>$$
\end{exmp}

\subsection{Equacions implícites de una varietat lineal}
\label{ss_eqimvl}

\begin{defn}
	
\end{defn}

\subsection{Combinacions lineals de punts}
\label{ss_comlip}

\begin{lem}
	$\sum\limits_{i=0}^{k}\lambda_{i}\vv{p q_{i}}=\left(\sum\limits_{i=0}^{k}\lambda_{i}\right)\vv{pr}+\sum\limits_{i=0}^{k}\lambda_{i}\vv{r q_{i}}$
\end{lem}
\begin{proof}
	$$\vv{pq}=\vv{pr}+\vv{rq}$$
	$$\vv{p q_{i}}=\vv{pr}+\vv{r q_{i}}$$
	$$\lambda_{i}\vv{p q_{i}}=\lambda_{i}\vv{pr}+\lambda_{i}\vv{r q_{i}}$$
	$$\sum_{i=0}^{k}\lambda_{i}\vv{p q_{i}}=\left(\sum_{i=0}^{k}\lambda_{i}\right)\vv{pr}+\sum_{i=0}^{k}\lambda_{i}\vv{r q_{i}}$$
\end{proof}

\begin{cor}
	Si $\sum\limits_{i=0}^k\lambda_{i}=0$, $\sum\limits_{i=0}^k\lambda_i\vv{pq_i}=\sum\limits_{i=0}^k\lambda_i\vv{rq_i}$, 
	definim $\sum\limits_{i=0}^k\lambda_iq_i=\sum\limits_{i=0}^k\lambda_i\vv{pq_i}$ si $\sum\limits_{i=0}^k\lambda_{i}=0$.\\
	Si $\sum\limits_{i=o}^k\lambda_i=1$, $\sum\limits_{i=0}^k\lambda_i\vv{pq_i}=\vv{pr}+\sum\limits_{i=0}^k\lambda_i\vv{rq_i}$, i per tant $p+\sum\limits_{i=0}^k\lambda_i\vv{pq_i}=r+\sum\limits_{i=0}^k\lambda_i\vv{rq_i}$,\\
	definim $\sum\limits_{i=0}^k\lambda_iq_i=p+\sum\limits_{i=0}^k\lambda_i\vv{pq_i}$ si $\sum\limits_{i=0}^k\lambda_{i}=1$.
\end{cor}

$\lambda_i\in K$ i $\sum\limits_{i=0}^k\lambda_i=x$, si $\left\{\begin{array}{ll}
x=0			& \text{llavors }\sum\lambda_ip_i\text{ és un vector.}				\\
x=1			& \text{llavors }\sum\lambda_ip_i\text{ és un punt.}				\\
x\neq 0\wedge x\neq 1	& \text{llavors }\frac{1}{\sum\lambda_i}\sum\lambda_ip_i\text{ és un punt.}	\\
\end{array}\right.$

\begin{exmp}(Baricentre de $m$ punts de $A$)\\
	Siguin $p_1,p_2,\dotsc ,p_m\in A$, es defineix el baricentre d'aquests punts per
	$$b=bar(p_1,\dotsc ,p_m))=\frac{1}{m}p_1+\dotsb +\frac{1}{m}p_m=\frac{p_1+\dotsb +p_m}{m}$$
	$$\sum\limits_{i=0}^m\frac{1}{m}=\frac{m}{m}=1\Rightarrow b\text{ és un punt de }A.$$
\end{exmp}
\begin{exmp}("Centre de masses" d'un sistema amb $k$ punts de $A$)\\
	Siguin $p_1,p_2,\dotsc ,p_k\in A$, i $m_1,\dotsc ,m_k\in K$, considerem els punts $p_i$ amb "masses" $m_i$ respectivament, llavors definim el "centre de masses" com el punt b tal que
	$$b=\frac{1}{\sum m_i}(\sum m_ip_i),\hspace{3em}\sum\frac{m_i}{\sum m_i}=1 \Leftrightarrow \sum m_i \neq 0$$
	$$\sum m_ip_i=(\sum m_i)b$$ 
\end{exmp}

\subsection{Independència afí}
\label{ss_indafi}

\begin{defn}
	Siguin $p_1,\dotsc ,p_m\in A$, diem que aquests punts son afinment independents si
	$$\sum\limits_{i=0}^m\lambda_i=0, \sum\limits_{i=0}^m\lambda_ip_i=0\Rightarrow\lambda_i=0,\:\forall i=1,\dotsc ,m$$
	I són afinment dependents en el cas contrari.
\end{defn}

\begin{lem}
	Les següents proposicions són equivalents:
	\begin{enumerate}
		\item $\sum\limits_{i=0}^m\lambda_i=0, \sum\limits_{i=0}^m\lambda_ip_i=0\Rightarrow\lambda_i=0,\:\forall i=1,\dotsc ,m$
		\item Si $\sum\limits_{i=0}^m\lambda_i=1,\:\sum\limits_{i=0}^m\mu_i=1$, i $\sum\limits_{i=0}^m\lambda_ip_i=\sum\limits_{i=0}^m\mu_ip_i\Rightarrow \lambda_i=\mu_i,\:\forall i$
		\item Si $\sum\limits_{i=2}^m\lambda_i\vv{p_1p_i}=0\Rightarrow \lambda_i=0,\:\forall i=2,\dotsc ,m$
	\end{enumerate}
\end{lem}
\begin{proof}
	$(1)\implies (2)$ \\
	$$\sum\lambda_ip_i=\sum\mu_ip_i,\:\sum\lambda_i=\sum\mu_i\implies \sum(\lambda_i-\mu_i)p_i=0,\:\sum(\lambda_i-\mu_i)=0$$
	$$\overset{(1)}{\implies}\lambda_i-\mu_i=0\implies\lambda_i=\mu_i$$
	$(2)\implies (1)$ \\
	Suposem
	$$\sum\lambda_ip_i=0\text{, amb }\sum\lambda_i=0$$
	Agafem un q diferent dels $p_i$
	$$\sum\lambda_ip_i+q=q,\:\sum\lambda_i+1=1\overset{(2)}{\implies}\lambda_i=0$$
\end{proof}

\begin{cor}
	Si $dim(A)=n$, el major nombre de punts afinment independents entre ells és $n+1$
\end{cor}

\begin{defn}
	Sigui $A$ un conjunt de $n+1$ punts afinment independents en $A^n$, s'anomena referència afí(o sistema de coordenades o coordenades baricèntriques).
\end{defn}

\begin{prop}
	Si ${p_0,\dotsc ,p_n}$ és una referència afí de $A$ y $q$ és qualsevol altre punt, llavors
	$$q=\sum\limits_{i=0}^n\lambda_ip_i\text{, amb }\sum\lambda_i=1$$
	i a més els $\lambda_i$ són únics. 
\end{prop}
\begin{proof}
	Considerem:
	$$\vv{p_0q}=\sum\limits_{i=0}^n\lambda_i\vv{p_0p_i},\:\lambda_i\in K$$
	llavors ${p_0,\dotsc ,p_n}$ és base de $E$, així:
	$$q-p_0=\sum\limits\lambda_i\vv{p_0p_i}=\sum\limits_{i=0}^n\lambda_i(p_i-p_0)$$
	$$q=(p_0-\sum\limits\lambda_ip_0)+\sum\limits\lambda_ip_i=(1-\sum\limits\lambda_i)p_0+\sum\limits\lambda_ip_i$$
\end{proof}

\begin{defn}
	Fixada la referència ${p_0,\dotsc ,p_n}$ de $A$, s'anomenen coordenades afins o baricèntriques de $q\in A$ als escalars $(\lambda_0, \dotsc ,\lambda_n)$ tals que:
	$$q=\sum_{i=0}^n\lambda_ip_i$$
\end{defn}
\begin{exmp}
	Siguin ${p_0,\dotsc ,p_n}$ punts afinment independent de $A$, el baricentre b té coordenades baricèntriques:
	$$b=(\frac{1}{n+1},\dotsc ,\frac{1}{n+1})$$
\end{exmp}

\begin{prop}
	Sigui $\mathbb{L}\subset A$ un subvarietat lineal, amb $dim(A)=n$ i $dim(\mathbb{L})=1$, per tant $\mathbb{L}$ és una recta. Siguin $P,Q\in\mathbb{L}$, amb $P\neq Q$, $(P,Q)$ és una referència afí de $\mathbb{L}$. \\
	A més, donat un altre punt $X\in A$ 
	$$X=\lambda P+(1-\lambda)Q\iff\vv{PX},\vv{PQ}\text{ són l.d.}$$
\end{prop}
\begin{exmp}
	Si $dim(A)=2$ i $\mathcal{R}$ és una referència cartesiana i $P=(p_1,p_2),\: Q=(q_1,q_2),\: X=(x_1,x_2)$ punts de $A$, sabem que:
	\[rang\left(\begin{array}{cc}
		x_1-p_1 & q_1-p_1 \\
		x_2-p_2 & q_2-p_2 \\
	\end{array}\right)=1\iff\left|\begin{array}{cc}
		x_1-p_1 & q_1-p_1 \\
		x_2-p_2 & q_2-p_2 \\
	\end{array}\right|=0\]
	\[\iff\left|\begin{array}{ccc}
		1   & 1   & 1   \\
		p_1 & x_1 & q_1 \\
		p_2 & x_2 & q_2 \\
	\end{array}\right|=0\iff\left|\begin{array}{ccc}
		p_0 & x_0 & q_0 \\
		p_1 & x_1 & q_1 \\
		p_2 & x_2 & q_2 \\
	\end{array}\right|=0\]
	si $p_0,p_1,p_2$ són les coordenades baricèntriques de P, i així respectivament pels tres punts.
\end{exmp}

\begin{defn}
	Siguin $a,b,c\in A$ 3 punts alineats amb $b\neq c$ i $\lambda\in K$, definim la raó simple com $(a,b,c):=\lambda$ tal que
	$$\vv{ac}=(a,b,c)\vv{bc}\iff\vv{ac}=\lambda\vv{bc}$$
\end{defn}

\begin{prop}
	Siguin $a,b,c\in A$ 3 punts alineats amb $b\neq c$ i raó simple $\lambda =(a,b,c)\in K$, podem expressar $a$ com:
	$$a=(a,b,c)b+(1-(a,b,c))c$$
\end{prop}
\begin{proof}
	Com $b$ i $c$ formen una referència afí de la recta que els uneix, podem expressar $a$ com
	\[a=\lambda b+(1-\lambda)c\]
	$$\begin{array}{rcl}
		\vv{ac} & = & c-a=c-(\lambda b+(1-\lambda )c)                      \\
		        & = & (\lambda c +(1-\lambda )c)-(\lambda b+(1-\lambda )c) \\
		        & = & \lambda (c-b)                                        \\
		        & = & \lambda \vv{bc}                                      \\
	\end{array}$$
	Per tant $\lambda =(a,b,c)$, i així:
	$$a=(a,b,c)b+(1-(a,b,c))c$$	
\end{proof}

\begin{prop}
	Sigui $A,B,C\in A$, $dim(A)=n$, i $A=(a_1,\dotsc ,a_n),B=(b_1,\dotsc ,b_n),C=(c_1,\dotsc ,c_n)$ les coordenades dels punts en un sistema de coordenades, la raó simple serà: \\
	\[(A,B,C)=\lambda =\frac{c_1-a_1}{c_1-b_1}=\dotsb =\frac{c_n-a_n}{c_n-b_n}\]
\end{prop}
\begin{proof}
	$(A,B,C)=\lambda\implies\vv{AC}=\lambda\vv{BC}$ \\
	\[\implies (c_1-a_1,\dotsc ,c_n-a_n)=\lambda (c_1-b_1,\dotsc ,c_n-b_n)\implies\lambda=\frac{c_1-a_1}{c_1-b_1}=\dotsb =\frac{c_n-a_n}{c_n-b_n}\]
\end{proof}

\subsection{Operacions amb subvarietats}
\label{ss_opsvar}

\begin{defn}(Intersecció de subvarietats lineals)
	Sigui $A$ un espai afí de espai director $E$, i $\mathbb{L}_1,\mathbb{L}_2\in A$ dos subvarietats lineals que podem expressar com $\mathbb{L}_1=p_1+F_1$ i $\mathbb{L}_2=p_2+F_2$, amb $p_1\in\mathbb{L}_1,\: p_2\in\mathbb{L}_2$ i $F_1,F_2\subseteq E$. Definirem la intersecció de dos subvarietats lineals com la subvarietat lineal més gran que conté nomès punts de les dues, i la denotarem per $\mathbb{L}_1\cap\mathbb{L}_2$ o $\mathbb{L}_1\land\mathbb{L}_2$.
\end{defn}

\begin{prop}
	Sigui $A$ un espai afí de espai director $E$, $\mathbb{L}_1,\mathbb{L}_2\in A$ dos subvarietats lineals de espais directors $F_1,F_2\in E$ respectivament, i $\mathbb{L}_1\cap\mathbb{L}_2\neq\varnothing$, $\mathbb{L}_1\cap\mathbb{L}_2$ és una subvarietat lineal de $A$ de espai director $F_1\cap F_2$.
\end{prop}
\begin{proof}
	\[\mathbb{L}_1\cap\mathbb{L}_2\neq\varnothing\implies\exists c\in A: c\in\mathbb{L}_1\land c\in\mathbb{L}_2\implies\left\{\begin{array}{rcl}
		\mathbb{L}_1 & = & c+F_1 \\
		\mathbb{L}_2 & = & c+F_2 \\
	\end{array}\right.\implies\mathbb{L}_1\cap\mathbb{L}_2=(c+F_1)\cap (c+F_2)\]
	\[ u_1\in F_1, u_2\in F_2, p\in \mathbb{L}_1\cap\mathbb{L}_2: \left\{\begin{array}{rcl}
		p & = & c+u_1 \\
		p & = & c+u_2 \\
	\end{array}\right\}\iff\mathbb{L}_1\cap\mathbb{L}_2=c+F_1\cap F_2\]
\end{proof}

\begin{prop}
	$\mathbb{L}_1\cap\mathbb{L}_2\neq\varnothing\iff\vv{p_1p_2}\in(F_1+F_2)$
\end{prop}
\begin{proof}
	$\implies )\hspace{2em}\mathbb{L}_1\cap\mathbb{L}_2\neq\varnothing\implies\exists c\in\mathbb{L}_1\cap\mathbb{L}_2\implies$
	\[\implies\left\{\begin{array}{rcll}
		c & = & p_1+u_1, & u_1\in F_1 \\
		c & = & p_2+u_2, & u_2\in F_2 \\
	\end{array}\right\}\implies p_1+u_1=p_2+u_2\implies u_1-u_2=p_2-p_1\]
	\[\left.\begin{array}{c}
		u_1-u_2\in F_1\cap F_2 \\
		u_1-u_2=p_2-p_1        \\
		p_2-p_1=\vv{p_1p_2}    \\
	\end{array}\right\}\implies\vv{p_1p_2}\in (F_1+F_2)\] \\
	$\impliedby )\hspace{2em}\vv{p_1p_2}\in (F_1+F_2)\implies\vv{p_1p_2}=p_2-p_1=u_1-u_2\implies p_2+u_2=p_1+u_1=c$
	\[\left.\begin{array}{r}
		p_1+u_1\in F_1    \\
		p_2+u_2\in F_2    \\
		p_2+u_2=p_1+u_1=c \\
	\end{array}\right\}\implies c\in\mathbb{L}_1\cap\mathbb{L}\implies\mathbb{L}_1\cap\mathbb{L}\neq\varnothing\]
\end{proof}

\begin{defn}(Paral·lelisme entre varietats lineals)
	Diem que dues subvarietats lineals tenen una relació de paral·lelisme, i per tant són paral·leles, si
	\[\mathbb{L}_1\parallel\mathbb{L}_2\iff (F_1\subseteq F_2)\lor (F_2\subseteq F_1)\]
\end{defn}

\begin{prop}
	La relació de paral·lelisme és de tipus:
	\begin{enumerate}
		\item Reflexiva: $\forall\mathbb{L}_1[\mathbb{L}_1\parallel\mathbb{L}_1]$
		\item Simètrica: $\forall\mathbb{L}_1,\forall\mathbb{L}_2 [\mathbb{L}_1\parallel\mathbb{L}_2\implies\mathbb{L}_2\parallel\mathbb{L}_1]$
	\end{enumerate}
	En canvi, no és transitiva, i per tant no és relació de equivalència.
\end{prop}
\begin{proof}\begin{enumerate}
	\item Reflexiva: $\mathbb{L}_1\parallel\mathbb{L}_1\iff (F_1\subseteq F_1)\lor (F_1\subseteq F_1)$
	\item Simètrica: $\mathbb{L}_1\parallel\mathbb{L}_2\iff (F_1\subseteq F_2)\lor (F_2\subseteq F_1)\implies\mathbb{L}_2\parallel\mathbb{L}_1$
	\end{enumerate}
	Contraxemple de transitivitat: $\mathbb{L}_1\parallel\mathbb{L}_2,\mathbb{L}_2\parallel\mathbb{L}_3\iff ((F_1\subseteq F_2)\lor (F_2\subseteq F_1))\land((F_2\subseteq F_3)\lor (F_3\subseteq F_2))$
	Si $F_2\subseteq F_3$ i $F_2\subseteq F_1\implies\mathbb{L}_1\parallel\mathbb{L}_2\land\mathbb{L}_2\parallel\mathbb{L}_3$ però $\mathbb{L}_1\nparallel\mathbb{L}_3$ ja que $(F_1\nsubseteq F_3)\lor (F_3\nsubseteq F_1)$.
\end{proof}

\begin{defn}(Suma de subvarietats lineals)
	Definim la suma de dos subvarietats lineals $\mathbb{L}_1$ i $\mathbb{L}_2$, com la subvarietat lineal més petita que les conté, i la denotem per $\mathbb{L}_1+\mathbb{L}_2$ o $\mathbb{L}_1\lor\mathbb{L}_2$.
\end{defn}

\begin{prop}
	$\mathbb{L}_1+\mathbb{L}_2=p_1+<\vv{p_1p_2}>+F_1+F_2$
\end{prop}
\begin{proof}
	Per definició, $p_1+<\vv{p_1p_2}>+F_1+F_2$ és una subvarietat lineal i conté $\mathbb{L}_1$ i $\mathbb{L}_2$:\\
	$\mathbb{L}_1=p_1+F_1\subseteq p_1+(<\vv{p_1p_2}>+F_1+F_2)$\\
	$\mathbb{L}_2=p_2+F_2\subseteq p_2+(<\vv{p_1p_2}>+F_1+F_2)$\\
	Falta comprovar que $p_1+<\vv{p_1p_2}>+F_1+F_2$ és la més petita que les conté:\\
	Definim un subvarietat lineal $\mathbb{M}=p_1+H=p_2+H$ tal que $\mathbb{L}_1,\mathbb{L}_2\subseteq\mathbb{M}$\\
	\[\left.\begin{array}{r}
		\mathbb{L}_1=p_1+F_1\subseteq p_1+H\implies F_1\subseteq H                    \\
		\mathbb{L}_2=p_2+F_2\subseteq p_2+H\implies F_2\subseteq H                    \\
		p_1,p_2\in\mathbb{M}\implies\vv{p_1p_2}\in H\implies <\vv{p_1p_2}>\subseteq H \\
	\end{array}\right\}\implies <\vv{p_1p_2}>+F_1+F_2\subseteq H\]
	\[\implies p_1+(<\vv{p_1p_2}>+F_1+F_2)\subseteq p_1+H=\mathbb{M}\]
	Per tant, $p_1+<\vv{p_1p_2}>+F_1+F_2$ conté qualsevol subvarietat lineal que contingui $\mathbb{L}_1$ i $\mathbb{L}_2$, i per definició és la suma de $\mathbb{L}_1$ i $\mathbb{L}_2$.
\end{proof}

\begin{exmp}
	$dim(A)=1,\mathbb{L}_1=\{p_1\},\mathbb{L}_2=\{p_2\}$, amb $p_1\neq p_2$ \\
	$\mathbb{L}_1\cap\mathbb{L}_2=\varnothing$ \\
	$\mathbb{L}_1+\mathbb{L}_2=\{p_1\}+\{p_2\}="p_1+p_2"=\{p_1\}\lor \{p_2\}=p_1\lor p_2=p_1+<\vv{p_1p_2}>=A$
\end{exmp}
\begin{exmp}
	$dim(A)=2,\mathbb{L}_1=\{p_1\},\mathbb{L}_2=p_2+<u>$, amb $\vv{p_1p_2}\neq\lambda u$ \\
	$\mathbb{L}_1\cap\mathbb{L}_2=\varnothing$ \\
	$\mathbb{L}_1+\mathbb{L}_2=p_1+<\vv{p_1p_2}>+<u>=A$
\end{exmp}

\subsection{Fòrmules de Grassman afins}
\label{ss_frgrss}

\begin{thm}\begin{enumerate}[label=\emph{\alph*})]
	\item Si $\mathbb{L}_1\cap\mathbb{L}_2\neq\varnothing$, llavors 
	\[dim(\mathbb{L}_1+\mathbb{L}_2)=dim(\mathbb{L}_1)+dim(\mathbb{L}_2)-dim(\mathbb{L}_1\cap\mathbb{L}_2)\]
	\item Si $\mathbb{L}_1\cap\mathbb{L}_2=\varnothing$, llavors
	\[dim(\mathbb{L}_1+\mathbb{L}_2)=dim(\mathbb{L}_1)+dim(\mathbb{L}_2)+1-dim(F_1\cap F_2)\]
\end{enumerate}\end{thm}
\begin{proof}
	$dim(\mathbb{L}_1+\mathbb{L}_2)=dim(p_1+<\vv{p_1p_2}>+F_1+F_2)=dim(<\vv{p_1p_2}>+F_1+F_2)=(\ast )$ \\
	\begin{enumerate}[label=\emph{\alph*})]
	\item $\mathbb{L}_1\cap\mathbb{L}_2\neq\varnothing\implies\vv{p_1p_2}\in F_1+F_2$ \\
	$(\ast )=dim(F_1+F_2)=dim(F_1)+dim(F_2)-dim(F_1\cap F_2)=dim(\mathbb{L}_1)+dim(\mathbb{L}_2)-dim(\mathbb{L}_1\cap\mathbb{L}_2)$
	\item $\mathbb{L}_1\cap\mathbb{L}_2=\varnothing\implies\vv{p_1p_2}\notin F_1+F_2$ \\
	$(\ast )=1+dim(F_1+F_2)=1+dim(F_1)+dim(F_2)-dim(F_1\cap F_2)$
\end{enumerate}\end{proof}

\begin{exmp}(Posició relativa d'un hiperpla i una recta a $A^n$)
	Siguin $\mathbb{L}_1$ un hiperpla i $\mathbb{L}_2$ una recta, per tant $dim(\mathbb{L}_1)=n-1$ i $dim(\mathbb{L}_2)=1$, i podem diferenciar dos casos:
	\begin{itemize}
		\item $\mathbb{L}_1\cap\mathbb{L}_2\neq\varnothing\implies\mathbb{L}_1\cap\mathbb{L}_2\subseteq\mathbb{L}_2\left\{\begin{array}{l}
			dim(\mathbb{L}_1\cap\mathbb{L}_2)=0\implies\mathbb{L}_1\cap\mathbb{L}_2=\{p\} \\
			dim(\mathbb{L}_1\cap\mathbb{L}_2)=1\implies\mathbb{L}_2\subseteq\mathbb{L}_1  \\
		\end{array}\right.$
		\item $\mathbb{L}_1\cap\mathbb{L}_2=\varnothing\implies dim(\mathbb{L}_1+\mathbb{L}_2)=n-1+1+1-dim(F_1\cap F_2)=n+1-dim(F_1\cap F_2)$ \\
		\[\left.\begin{array}{r}
			dim(\mathbb{L}_1+\mathbb{L}_2)=n+1-dim(F_1\cap F_2) \\
			dim(\mathbb{L}_2)=1\implies dim(F_2)=1 \\
			\mathbb{L}_1+\mathbb{L}_2\subseteq A\implies dim(\mathbb{L}_1+\mathbb{L}_2)\leq n \\
		\end{array}\right\}\implies dim(F_1\cap F_2)=1\implies F_2=F_1\cap F_2\] \\
		$\implies F_2\subseteq F_1\implies \mathbb{L}_1\parallel\mathbb{L}_2$
	\end{itemize}
\end{exmp}

\subsection{Teoremes clàssics}
\label{ss_teocla}

\begin{thm}[de Tales]
	Siguin $r,s$ dos rectes en un pla afí i $l_1,l_2,l_3$ 3 rectes paral·leles i que tallen a $r$ en $p_1,p_2,p_3$, i a $s$ en $q_1,q_2,q_3$, respectivament; llavors $(p_1,p_2,p_3)=(q_1,q_2,q_3)$.
\end{thm}
\begin{proof}
	Suposant $p_1\neq q_1$, definim la referència $\mathcal{R}=\{p_1,\vv{p_1p_2},\vv{p_1q_1}\}$, i expressem els punts $p_i$ i $q_i$ en aquesta referència: \\
	\[\begin{array}{ll}
		p_1=(0,0) & q_1=(0,1)\\
		p_2=(1,0) & q_2=p_1+\vv{p_1p_2}+\vv{p_2q_2}=p_1+\vv{p_1p_2}+b\vv{p_1q_1}=(1,b) \\
		p_3=(a,0) & q_3=p_1+\vv{p_1p_3}+\vv{p_3q_3}=p_1+a\vv{p_1p_2}+c\vv{p_1q_1}=(a,c)\\
	\end{array}\]
	\[\left.\begin{array}{l}
		(p_1,p_2,p_3)=\frac{a-0}{a-1}=\frac{0-0}{0-0} \\
		(q_1,q_2,q_3)=\frac{a-0}{a-1}=\frac{c-1}{c-b} \\
	\end{array}\right\}\implies (p_1,p_2,p_3)=(q_1,q_2,q_3)\]
\end{proof}

\begin{thm}[de Menelao]
	Siguin $A_1,A_2,A_3$ 3 punts afinment independents en un pla afí, i sigui $l$ una recta que talla amb els costats $a_1=\{A_2\}+\{A_3\},a_2=\{A_3\}+\{A_1\},a_3=\{A_1\}+\{A_2\}$ del triangle que formen $A_1,A_2,A_3$, en els punts $B_1,B_2,B_3$, respectivament; llavors
	\[(A_1,A_2,B_3)(A_2,A_3,B_1)(A_3,A_1,B_2)=1\]
\end{thm}
\begin{proof}
	$\mathcal{R}=\{A_1,\vv{A_1A_2},\vv{A_1A_3}\}$ \\
	\[\begin{array}{ll}
		A_1=(0,0) & B_1=(x,y)=\lambda B_3+(1-\lambda )B_2=(\lambda a,(1-\lambda )b) \\
		A_2=(1,0) & B_2=(0,b) \\
		A_3=(0,1) & B_3=(a,0) \\
	\end{array}\]
	\[\left.\begin{array}{l}
		x+y=1 \\
		(x,y)=(\lambda a,(1-\lambda )b) \\
	\end{array}\right\}\implies\lambda a+(1-\lambda )b\implies\lambda =\frac{1-b}{a-b}\text{, amb }a\neq b\]
	\[B_1=\left(\frac{1-b}{a-b}a,\left(1-\frac{1-b}{a-b}\right)b\right)=\left(\frac{1-b}{a-b}a,\frac{a-b-1+b}{a-b}b\right)=\left(\frac{1-b}{a-b}a,\frac{a-1}{a-b}b\right)\] \\
	$\displaystyle{(A_1,A_2,B_3)=\frac{a}{a-1}}$ \\
	$\displaystyle{(A_2,A_3,B_1)=\frac{\frac{1-b}{a-b}a-1}{\frac{1-b}{a-b}a}=\frac{a-ab-a+b}{a-ab}=\frac{b(1-a)}{a(1-b)}}$ \\
	$\displaystyle{(A_3,A_1,B_2)=\frac{b-1}{b}}$ \\
	\[(A_1,A_2,B_3)(A_2,A_3,B_1)(A_3,A_1,B_2)=\left(\frac{a}{a-1}\right)\left(\frac{b(1-a)}{a(1-b)}\right)\left(\frac{b-1}{b}\right)=1\]
\end{proof}

\begin{thm}[de Ceva]
	Siguin $A_1,A_2,A_3$ 3 punts afinment independents en un pla afí, i sigui $P$ un punt del pla
	\[\begin{array}{l}
		B_1=(\{A_2\}+\{A_1\})\cap (\{A_1\}+\{P\}) \\
		B_2=(\{A_3\}+\{A_1\})\cap (\{A_2\}+\{P\}) \\
		B_3=(\{A_1\}+\{A_2\})\cap (\{A_3\}+\{P\}) \\
	\end{array}\]
	Llavors
	\[(A_1,A_2,B_3)(A_2,A_3,B_1)(A_3,A_1,B_2)=-1\]
\end{thm}
\begin{proof}
	$\mathcal{R}=\{A_1,\vv{A_1A_2},\vv{A_1A_3}\}$ \bigskip \\ 
	$\begin{array}{rlrl}
		P   & =(a,b)\hspace{2em} & & \\
		A_1 & =(0,0)\hspace{2em} & B_1 & =(x,y)\text{, amb }x+y=1\implies B_1=(x,1-x) \\
		A_2 & =(1,0)\hspace{2em} & B_2 & =(0,d) \\
		A_3 & =(0,1)\hspace{2em} & B_3 & =(c,0) \\
	\end{array}$ \bigskip  \\
	$\displaystyle{\mu_1=(B_1,P,A_1)=\frac{\vv{B_1A_1}}{\vv{PA_1}}=\frac{x}{a}=\frac{x-1}{-b}\implies\frac{-b}{a}=\frac{x-1}{x}}$ \\
	$\displaystyle{\mu_2=(B_2,P,A_2)=\frac{\vv{B_2A_2}}{\vv{PA_2}}=\frac{1}{1-a}=\frac{d}{b}\implies d=\frac{b}{1-a}}$ \\
	$\displaystyle{\mu_3=(B_3,P,A_3)=\frac{\vv{B_3A_3}}{\vv{PA_3}}=\frac{1}{1-b}=\frac{c}{a}\implies c=\frac{a}{1-b}}$ \bigskip \\
	$\displaystyle{\lambda_1=(A_1,A_2,B_3)=\frac{\vv{A_1B_3}}{\vv{A_2B_3}}=\frac{c}{c-1}=\frac{\left(\frac{a}{1-b}\right)}{\left(\frac{a}{1-b}\right)-1}=\frac{a}{a+b-1}}$ \\
	$\displaystyle{\lambda_2=(A_2,A_3,B_1)=\frac{\vv{A_2B_1}}{\vv{A_3B_1}}=\frac{x-1}{x}=\frac{-b}{a}}$ \\
	$\displaystyle{\lambda_3=(A_3,A_1,B_2)=\frac{\vv{A_3B_2}}{\vv{A_1B_2}}=\frac{d-1}{d}=\frac{\left(\frac{b}{1-a}\right)-1}{\left(\frac{b}{1-a}\right)}=\frac{a+b-1}{b}}$ \bigskip \\
	\[\lambda_1\lambda_2\lambda_3=\left(\frac{a}{a+b-1}\right)\left(\frac{-b}{a}\right)\left(\frac{a+b-1}{b}\right)\]
\end{proof}

\subsection{Aplicacions afins}
\label{ss_aplafi}

Siguin $\mathbb{A},\mathbb{A}_1,\dotsc ,\mathbb{A}_n$ espais afins de espai director $E,E_1,\dotsc ,E_n$ respectivament

\begin{defn}
	Una aplicació afí $f:\mathbb{A}_1\to\mathbb{A}_2$ és una aplicació de conjunts tal que $\exists$ una aplicació lineal
	\[\widetilde{f}:E_1\to E_2\]
	que verifica
	\[\forall p\in\mathbb{A}_1,\forall\vv{u}\in E_1[f(p+\vv{u})=f(p)+\widetilde{f}(\vv{u})]\]
\end{defn}

\begin{prop}
	$\exists\widetilde{f}:f(p+\vv{u})=f(p)+\widetilde{f}(\vv{u})\implies\widetilde{f}$ és única, i en efecte
	\[\forall\vv{u}\in E_1[\widetilde{f}=f(p+\vv{u})-f(p)]\]
	i així $\widetilde{f}$ queda determinada per $f$, i l'anomenem l'aplicació lineal associada a $f$.
\end{prop}

Altres formes d'expressar-ho: \\
Si $q=p+\vv{u}=p+\vv{pq}, \left.\begin{array}{rl}
	f(q)= & f(p+u)=f(p)+\widetilde{f}(\vv{pq}) \\
	f(q)= & f(p)+\vv{f(p)f(q)} \\
\end{array}\right\}\implies\widetilde{f}(\vv{pq})=\vv{f(p)f(q)}$ \bigskip \\
Usant els axiomes dels espais afins: \\
Sabem que $\forall p\in\mathbb{A}_1$, la aplicació
\[\begin{array}{rcl}
	\phi_{p} :E_1	& \to	& \mathbb{A}_1 	\\
	\vv{v}		& \mapsto		& q+\vv{v}	\\
\end{array}\]
és bijectiva. \\
\[\forall\vv{u}\in E_1[f(\phi_{p}(\vv{u}))=\phi_{f(p)}(\widetilde{f}(\vv{u}))\implies (f\circ\phi_{p})(\vv{u})=(\phi_{f(p)}\circ\widetilde{f})(\vv{u})]\] \\
\begin{tikzpicture}
  \matrix (m) [matrix of math nodes,row sep=3em,column sep=4em,minimum width=2em]
  {
     E_1 & \mathbb{A}_1 \\
     E_2 & \mathbb{A}_2 \\};
  \path[-stealth]
    (m-1-1) edge node [left] {$\widetilde{f}$} (m-2-1)
            edge node [above] {$\phi_p$} (m-1-2)
    (m-2-1.east|-m-2-2) edge node [below] {$\phi_{f(p)}$}
            node [above] {} (m-2-2)
    (m-1-2) edge node [right] {$f$} (m-2-2);
\end{tikzpicture} \\
$f\circ\phi_p=\phi_{f(p)}\circ\widetilde{f}$
\begin{cor}
	$f=\phi_{f(p)}\circ\widetilde{f}\circ\phi_p^{-1}$ \\
	$\widetilde{f}=\phi_{f(p)}^{-1}\circ f\circ\phi_p$ \\
	i així:
	\[\begin{array}{lcl}
		f\text{ bijectiva} & \iff & \widetilde{f}\text{ bijectiva} \\
		f\text{ injectiva} & \iff & \widetilde{f}\text{ injectiva} \\
		f\text{ exhaustiva} & \iff & \widetilde{f}\text{ exhaustiva} \\
	\end{array}\]
	i per tant, $f$ determina $\widetilde{f}$, i $\widetilde{f}$ junt amb un punt $p$ i la seva imatge $f(p)$ determina $f$.
\end{cor}

\begin{exmp}
	\[\begin{array}{rcl}
		id:\mathbb{A}	& \to	& \mathbb{A} 	\\
		p	& \mapsto		& id(p)=p	\\
	\end{array}\] \\
	$id(p+\vv{u})=p+\vv{u}=id(p)+\vv{u}\implies\widetilde{id}(\vv{u})=\vv{u}$ \\
	Per tant la identitat de $\mathbb{A}$ és afí.
\end{exmp}
\begin{exmp}
	Si $f$ i $g$ són afins, $g\circ f$ és afí i $\widetilde{g\circ f}=\widetilde{g}\circ\widetilde{f}$ \\
	En efecte, $\mathbb{A}_1\mathop{\to}\limits^f\mathbb{A}_2\mathop{\to}\limits^g\mathbb{A}_3$ \\
	\[(g\circ f)(p+\vv{u})=g(f(p+\vv{u}))=g(f(p)+\widetilde{f}(\vv{u}))=(g\circ f)(p)+(\widetilde{g}\circ \widetilde{f})(\vv{u})\implies \widetilde{g\circ f}=\widetilde{g}\circ\widetilde{f}\]
\end{exmp}

\begin{defn}
	Sigui $w\in E$, $w\neq 0$, definim
	\[\begin{array}{rcl}
		z_{\vv{w}}:\mathbb{A} & \to & \mathbb{A} \\
		p & \mapsto & p+\vv{w} \\
	\end{array}\]
\end{defn}

\begin{prop}
	$z_{\vv{w}}$ és una aplicació afí.
\end{prop}
\begin{proof}
	$z_{\vv{w}}(p+\vv{u})=p+\vv{u}+\vv{w}=p+\vv{w}+\vv{u}=z_{\vv{w}}(p)+\vv{u}=z_{\vv{w}}(p)+id_E(\vv{u})\implies z_{\vv{w}}$ és afí i $\widetilde{z_{w}}=id_E$
\end{proof}

\begin{prop}
	Si $f$ és afí i $\widetilde{f}=id$, llavors $f$ és una translació.
\end{prop}
\begin{proof}
	En efecte, considerem un punt $p$, i la seva imatge $f(p)$. \\
	Llavors $f(p)=p+\vv{w}\implies\vv{w}=\vv{pf(p)}$. \\
	Si ara agafem un altre punt $q=p+\vv{pq}$ tindrem
	\[f(q)=f(p)+\widetilde{f}(\vv{pq})=f(p)+\vv{pq}=p+\vv{w}+\vv{pq}=q+\vv{w}\implies\forall q,f=z_{\vv{w}}\]
\end{proof}

\begin{prop}
	$f,g$ translacions $\implies g\circ f$ translació
\end{prop}
\begin{proof}
	$\forall p[f(p)=p+\vv{w_1}],\forall p[g(p)=p+\vv{w_2}]$ \\
	$(g\circ f)(p)=g(f(p))=g(p+\vv{w_1})=(p+\vv{w_1})+\vv{w_2}=p+(\vv{w_1}+\vv{w_2})\implies g\circ f$ és la translació de vector $\vv{w_1}+\vv{w_2}$
\end{proof}

\begin{prop}
	Les translacions amb l'identitat i la composició d'aplicacions formen un grup.
\end{prop}
\begin{proof}
	$(z_{-w}\circ z_w)(p)=p$\\
	FALTA ACABAR
\end{proof}

\begin{defn}
	Sigui $r\neq 0,1$ i sigui $O$,$p$ dos punt de $\mathbb{A}$. Definim l'homotècia de centre $O$ i raó $r$ com
	\[\begin{array}{rccc}
		f: & \mathbb{A} & \to &\mathbb{A} \\
		& p=O+\vv{Op} & \mapsto & f(p)=O+r\vv{Op} \\
	\end{array}\]
	Si $r>1$ diem que $f$ és una dilatació. \\
	Si $1>r>0$ diem que $f$ és una contracció \\
	Si $r=-1$ diem que $f$ és una simetria central
\end{defn}

\begin{prop}
	$f$ és una homotècia $\iff$ és una aplicació afí amb $\widetilde{f}=r\cdot id_E$.
\end{prop}
\begin{proof}
	$\implies )\hspace{2em}f(p+\vv{u})=O+r\vv{O(p+\vv{u})}=O+r(\vv{Op}+\vv{u})=O+r\vv{Op}+r\vv{u}=f(p)+r\vv{u}\implies f$ és afí i $\widetilde{f}=r\cdot id_E$ \bigskip \\
	$\impliedby )\hspace{2em}f$ és afí i $\widetilde{f}=r\cdot id_E\implies $FALTA ACABARLO
\end{proof}

\begin{prop}
	El centre de la homotècia queda determinat per un punt $p$ i la seva imatge $f(p)$.
\end{prop}
\begin{proof}
	\[f(p)=O+rp-rO=(1-r)O+rp\implies -rp+f(p)=(1-r)O\implies O=\frac{-r}{1-r}p+\frac{1}{1-r}f(p)\]
\end{proof}

\begin{prop}
	El centre $O$ de una homotècia verifica $f(O)=O$, i per tant és un punt fix.
\end{prop}
\begin{proof}
	\[O=\frac{-r}{1-r}p+\frac{1}{1-r}p-\frac{1}{1-r}p+\frac{1}{1-r}f(p)=p+\frac{1}{1-r}\vv{pf(p)}\]
	\[f(O)=f(p)+\frac{r}{1-r}\vv{pf(p)}=\frac{1-r}{1-r}f(p)+\frac{r}{1-r}f(p)-\frac{r}{1-r}p=\frac{1}{1-r}f(p)-\frac{r}{1-r}p=O\]
\end{proof}

Donat un $p$ qualsevol, $p=O+\vv{Op}$, per tant $f(p)=f(O+\vv{Op})=f(O)+\widetilde{f}(\vv{Op})=O+r\vv{Op}$

\begin{prop}
	$f,g$ homotecies amb el mateix origen $O\implies g\circ f$ homotècia
\end{prop}
\begin{proof}
	$f(p)=O+r\vv{Op},g(p)=O+s\vv{Op}$ \\
	$(g\circ f)(p)=g(f(p))=g(O+r\vv{Op})=g(O)+sr\vv{Op}=O+sr\vv{Op}\implies g\circ f$ és una homotècia de centre $O$ i raó $rs$
\end{proof}

\begin{prop}
	Les homotècies de mateix origen formen un grup amb l'identitat i la composició.
\end{prop}
\begin{proof}
	FALTA ACABAR
\end{proof}

\begin{prop}
	$f,g$ homotecies tals que $f(x)=P+r\vv{Px},g(x)=Q+s\vv{Qx}$ \\ 
	\[\left\{\begin{array}{l}
		rs=1\implies g\circ f\text{ és una translació de vector de translació} w=(s-1)\vv{QP} \\
		rs\neq 1\implies g\circ f\text{ és una homotècia} \\
	\end{array}\right.\]
\end{prop}
\begin{proof}
	$f(x)=P+r\vv{Px},g(x)=Q+s\vv{Qx}$ \bigskip \\
	$(g\circ f)(x)=g(P+r\vv{Px})=Q+s\vv{Q(P+r\vv{Px})}=Q+s(P+r\vv{Px})-sQ=Q+s\vv{QP}+sr\vv{Px}=$ FALTA ACABARLO \\
	$\implies\widetilde{g\circ f}=\widetilde{g}\circ\widetilde{f}=srid_E$ \\
	$\implies\left\{\begin{array}{l}
		rs=1\implies g\circ f\text{ és una translació} \\
		rs\neq 1\implies g\circ f\text{ és una homotècia} \\
	\end{array}\right.$
	Si $g\circ f$ és una translació amb $rs=1$, el seu vector de translació serà \\
	$(g\circ f)(P)=g(P)=Q+s\vv{QP}$ \\
	Així el vector de la translació que busquem és
	$w=(g\circ f)(P)-P=Q+s\vv{QP}-P=-\vv{QP}+s\vv{QP}=(s-1)\vv{QP}$
\end{proof}

\begin{defn}
	Sigui $\mathbb{L}=p+F$ una subvarietat lineal de $\mathbb{A}$, i tenim una descomposició $E=F\oplus G$, llavors un punt $x\in \mathbb{A}$
	\[x=p+\vv{px}=p+\vv{px_F}+\vv{px_G}\]
	Definim la simetria respecte de $\mathbb{L}$ i en la direcció $G$ com
	\[s(x)=p+\vv{px_F}-\vv{px_G}\]
\end{defn}

\begin{prop}\begin{enumerate}
	\item $x\in\mathbb{L}\implies s(x)=x$
	\item $\forall x, s\neq id[s^2(x)=x]\implies s^2=id\implies \widetilde{s}^2=id$
\end{enumerate}\end{prop}
\begin{proof}\begin{enumerate}
	\item FALTA ACABARLO
	\item $s(s(x))=s(p+\vv{px_F}-\vv{px_G})=p+\vv{px_F}+\vv{px_G}=x$
\end{enumerate}\end{proof}

\begin{defn}
	Sigui $x=p+\vv{px}=p+\vv{px_F}+\vv{px_G}$, definim la projecció sobre $\mathbb{L}$ en la direcció de $G$ com
	\[\pi (x)=p+\vv{px_F}\]
\end{defn}

\begin{prop}
	$x\in\mathbb{L}\implies\pi (x)=x$
\end{prop}
\begin{proof}
	FALTA ACABAR
\end{proof}

\begin{defn}
	Una aplicació afí diem que és una afinitat si $f$ és bijectiva(equivalentment $\widetilde{f}$ és bijectiva).
\end{defn}

\begin{prop}
	Siguin $\mathbb{A}_1,\mathbb{A}_2$ espais afins, $p_1\in\mathbb{A}_1,p_2\in\mathbb{A}_2, h:E_1\to E_2$ una aplicació lineal. Llavors $\exists !$ aplicació afí $f:\mathbb{A}_1\to\mathbb{A}_2$ tal que $f(p_1)=p_2$ i $\widetilde{f}=h$.
\end{prop}
\begin{proof}
	Existència de $f$: \\
	Definim $f$ per
	\[\forall x\in\mathbb{A}_1[f(x)=f(p_1+\vv{p_1x})=f(p_1)+h(\vv{p_1x})=p_2+h(\vv{p_1x})]\]
	Veiem que $f$ és afí
	\[f(x+\vv{u})=p_2+h(\vv{p_1(x+\vv{u})})=p_2+h(\vv{p_1x}+\vv{u})=p_2+h(\vv{p_1x})+h(\vv{u})=f(x)+h(\vv{u})\]
	Llavors $f$ és afí i $\widetilde{f}=h$
\end{proof}

\begin{cor}1. 
	Si $p_1\in\mathbb{A}_1, p_2\in\mathbb{A}_2,$ i $e_1,\dotsc ,e_n$ és una base de $E_1$ i $w_1,\dotsc ,w_n$ $n$ vectors de $E_2$, llavors $\exists !$ aplicació afí $f:\mathbb{A}_1\to\mathbb{A}_2$ tal que $f(p_1)=p_2,\widetilde{f}(e_i)=w_i,i=1,\dotsc ,n$
\end{cor}
\begin{cor}2. 
	Si $p_0,p_1,\dotsc ,p_n$ són $n+1$ punts afinment indep de $\mathbb{A}_1,dim(\mathbb{A}_1)=n$, i $q_0,q_1,\dotsc ,q_n$ $n+1$ punts de $\mathbb{A}_2$, llavors $\exists !$ aplicació afí $f:\mathbb{A}_1\to\mathbb{A}_2$ tal que $f(p_i)=q_i,i=0,\dotsc ,n$
\end{cor}
\begin{proof}
	Siguin $p_i=p_0+\vv{p_0p_i}$, com els punts $p_i$ són afinment independent, els $n$ vectors $e_i=\vv{p_0p_i}$ són l.i. \\
	Per tant formen una base 
	\[f(p_i)=q_i=q_0+\vv{q_0q_i}=f(p_0+\vv{p_0p_i})=q_0+\widetilde{f}(\vv{p_0p_i})\]
	Després apliquem el cor.1 amb $\vv{e_i}=\vv{p_0p_i}$ i $\vv{w_i}=\vv{q_0q_i}$.
\end{proof}

\begin{prop}
	Si $f:\mathbb{A}_1\to\mathbb{A}_2$ és afí llavors
	\begin{itemize}
		\item $f(\sum\limits_{i=1}^r\lambda_ip_i)=\sum\limits_{i=1}^r\lambda_if(p_i)$, si $\sum\limits_{i=1}^r\lambda_i=1$
		\item $\widetilde{f}(\sum\limits_{i=1}^r\lambda_ip_i)=\sum\limits_{i=1}^r\lambda_if(p_i)$, si $\sum\limits_{i=1}^r\lambda_i=0$
	\end{itemize}
\end{prop}
\begin{proof}
	$f(\sum\lambda_ip_i)=f(\sum\lambda_ip_i-\sum\lambda_ip_0+p_0)=f(\sum\lambda_i\vv{p_0p_i}+p_0)=f(p_0)+\widetilde{f}(\sum\lambda_i\vv{p_0p_i})=f(p_0)+\sum\lambda_i\widetilde{f}(\vv{p_0p_i})=f(p_0)+\sum\lambda_i\vv{f(p_0)f(p_i)}=f(p_0)+\sum\lambda_i(f(p_i)-f(p_0))=\sum\lambda_if(p_i)$
	FALTA ACABAR
\end{proof}

\begin{cor}
	Siguin $p_1,p_2,p_3$ 3 punts alineats de $\mathbb{A}_1, p_2\neq p_3$, i sigui $f:\mathbb{A}_1\to\mathbb{A}_2$ una aplicació afí, llavors
	\begin{itemize}
		\item Si $f(p_2)\neq f(p_3),f(p_1)$ està sobre la recta que defineixen $f(p_2)$ i $f(p_3)$, i a més $(p_1,p_2,p_3)=(f(p_1),f(p_2),f(p_3))$
		\item Si $f(p_2)=f(p_3), f(p_1)=f(p_2)=f(p_3)$
	\end{itemize}
\end{cor}
\begin{proof}
	$p_1=\lambda p_2+(1-\lambda )p_3\implies f(p_1)=\lambda f(p_2)+(1-\lambda )f(p_3)$ \\
	$f(p_2)=f(p_3)\implies f(p_1)=f(p_2)$
\end{proof}

\subsection{Expressió d'una aplicació afí en coord. cartesianes}
\label{ss_apcoca}

Sigui $\mathbb{A}_1$ un espai afí amb un sistema de cordenades $\mathcal{R}_1=\{O_1;e_1,dotsc ,e_n\},\mathcal{R}_2=\{O_2;f_1,dotsc ,f_m\}$. Si $f:\mathbb{A}_1\to\mathbb{A}_2$ és una aplicació afí amb una aplicació lineal associada $\widetilde{f}$ té una matriu $M=(a_j^i)$ en aquestes bases.
\[\widetilde{f}(\vv{e_i})=\sum\limits_{j=1}^ma_i^jf_j,i=1,\dotsc ,n\]
\[M=\left(\begin{array}{cccc}
	a_1^1&a_2^1&\cdots&a_n^1\\
	a_1^2&a_2^2& &\vdots\\
	\vdots& &\ddots&\vdots\\
	a_1^m&\cdots&\cdots&a_n^m\\
\end{array}\right)\hspace{2em}X=\left(\begin{array}{c}
	x^1\\
	\vdots\\
	x^n\\
\end{array}\right)\]
Si $x\in E_1$, llavors $x=\sum x^i\vv{e_i}$, i així $\widetilde{f}(x)=\sum\widetilde{f}(\vv{e_i})x^i=MX$ \\
Si $P$ és un punt de $\mathbb{A}_1$
\[P=O_1+\vv{O_1P}=O_1+\sum\limits_{i=1}^nx^i\vv{e_i}\]
aquestes $x^i$ son les coordenades de $P$ en la referència $\mathcal{R}_1$ \\
Apliquem $f$ a $P$, tenim
\[f(P)=f(O_1)+\widetilde{f}(\vv{O_1P})=O_2+\sum\limits_{j=1}^mb^jf_j+MX\]
Llavors $(b_j)$ són les coordenades de $f(O_1)$ en $\mathcal{R}_2$
En forma matricial resulta
\begin{itemize}
	\item Si $X$ és el vector columna de coordenades de $P$
	\item Si $B$ és el vector columna de coordenades de $f(O_1)$
	\item Si $Y$ és el vector columna de coordenades de $f(P)$
\end{itemize}
\[Y=B+MX\iff\left(\begin{array}{c}
	y^1\\
	\vdots\\
	y^m\\
\end{array}\right)=\left(\begin{array}{c}
	b^1\\
	\vdots\\
	b^m\\
\end{array}\right)+\left(\begin{array}{ccc}
	a_1^1&\cdots&a_n^1\\
	\vdots&\ddots&\vdots\\
	a_1^m&\cdots&a_n^m\\
\end{array}\right)\left(\begin{array}{c}
	x^1\\
	\vdots\\
	x^n\\
\end{array}\right)\]
Això equival a la equació matricial
\[\left(\begin{array}{c}
	y^1\\
	\vdots\\
	y^m \\ \hline
	1\\
\end{array}\right)=\left(\begin{array}{ccc|c}
	a_1^1&\cdots&a_n^1&b^1\\
	\vdots&\ddots&\vdots&\vdots\\
	a_1^m&\cdots&a_n^m&b^m\\ \hline
	0&0&0&1\\
\end{array}\right)\left(\begin{array}{c}
	x^1\\
	\vdots\\
	x^n\\ \hline
	1\\
\end{array}\right)\]

\subsection{Matriu d'una aplicació afí}
\label{ss_maapaf}

Sigui $\mathbb{A}_1,\mathbb{A}_2$ espais afins de referecies $\mathcal{R}_1=\{O_1;e_1,\dotsc ,e_n\},\mathcal{R}_2=\{O_2;e_i',\dotsc ,e_n'\}$ amb una aplicació afí $f:\mathbb{A}_1\to\mathbb{A}_2,\widetilde{f}:E_1\to E_2$ aplicació lineal, la matriu de $f$ serà
\[M(f)=\left(\begin{array}{ccc|c}
	\ddots&&\iddots&\vdots\\
	&M(\widetilde{f})&&f(O_1)\\
	\iddots&&\ddots&\vdots\\ \hline
	0&\cdots&0&1\\
\end{array}\right)\hspace{2em}M(\widetilde{f})=\left(\begin{array}{c|c|c}
	\vdots&\vdots&\vdots\\
	\widetilde{f}(e_1)&\widetilde{f}(e_2)&\widetilde{f}(e_3)\\
	\vdots&\vdots&\vdots\\
\end{array}\right)\]

\begin{exmp}
	$f$ translació de vector $w=w_1,\dotsc,w_n$
	\[M(f)=\left(\begin{array}{ccc|c}
		1&&0&w_1\\
		&\ddots&&\vdots\\
		0&&1&w_n\\ \hline
		0&\cdots&0&1\\
	\end{array}\right)\]
\end{exmp}
\begin{exmp}
	$f$ homotècia de raó $r$ i centre $O'$, amb la referència $\mathcal{R}_1=\{O_1;e_1,\dotsc ,e_n\}$
	\[M(f)=\left(\begin{array}{ccc|c}
		r&&0&\vdots\\
		&\ddots&&f(O_1)\\
		0&&r&\vdots\\ \hline
		0&\cdots&0&1\\
	\end{array}\right)\]
\end{exmp}
\begin{exmp}
	Simetria respecte a $\mathbb{L}=a+F$ i en la direcció de $G$, amb $E=F\oplus G, s(a+\vv{u})=a+\vv{u_f}-\vv{u_G}$, amb la referència $\mathcal{R}_1=\{O_1;e_1,\dotsc ,e_r,e_{r+1},\dotsc,en\}$ amb $\{e_1,\dotsc,e_{r+1}\}$ base de $F$ i $\{e_{r+1},\dotsc,e_n\}$ base de $G$
	\[M(s)=\left(\begin{array}{ccc|ccc|c}
		1&&0&&&&0\\
		&\ddots&&&0&&\vdots\\ 
		0&&1&&&&\vdots\\ \hline
		&&&-1&&0&\vdots\\
		&0&&&\ddots&&\vdots\\
		&&&0&&-1&0\\ \hline
		0&\cdots&\cdots&\cdots&\cdots&0&1\\
	\end{array}\right)\]
\end{exmp}
\begin{exmp}
	Projecció sobre $\mathbb{L}=a+F$ en la direcció de $G$ amb $E=F\oplus G,\pi(a+\vv{u})=a+\vv{u_F}$, en la referència anterior tindrem
	\[M(\pi)=\left(\begin{array}{ccc|ccc|c}
		1&&0&&&&0\\
		&\ddots&&&0&&\vdots\\ 
		0&&1&&&&\vdots\\ \hline
		&&&&&&\vdots\\
		&0&&&0&&\vdots\\
		&&&&&&0\\ \hline
		0&\cdots&\cdots&\cdots&\cdots&0&1\\
	\end{array}\right)\]
\end{exmp}

\begin{lem}
	Si $f:\mathbb{A}_1\to\mathbb{A}_2,g:\mathbb{A}_2\to\mathbb{A}_3$ aplicacions afins, amb referencies $\mathcal{R}_1,\mathcal{R}_2,\mathcal{R}_3$ per als espais afins $\mathbb{A}_1,\mathbb{A}_2,\mathbb{A}_3$ respectivament. Llavors
	\[M(g\circ f)=M(g)M(f)\]
\end{lem}
\begin{proof}
	FALTA ACABARLO
\end{proof}

\begin{defn}[Punts fixos d'una aplicació afí]
	Sigui $f:\mathbb{A}\to\mathbb{A}$ una aplicació afí, diem que un punt $p\in\mathbb{A}$ és un punt fix si $f(p)=p$. Anomenem $\Gamma_f$ al conjunt de tots el punts fixos de $f$.
	\[\Gamma_f=\{x\in\mathbb{A}:f(x)=x\}\subseteq\mathbb{A}\]
\end{defn}

\begin{prop}1.
	Si $\Gamma_f\neq\varnothing$, llavors $\Gamma_f$ és una subvarietat lineal de $\mathbb{A}$ de espai director $Ker(\widetilde{f}-1)$, així $\Gamma_f=p+Ker(\widetilde{f}-1)$, amb $f(p)=p$, i per tant $dim(\Gamma_f)=dim(Ker(\widetilde{f}-1))$
\end{prop}
\begin{proof}
	$\Gamma_f\neq\varnothing\implies\exists p:f(p)=p$ \\
	Qualsevol altre $x\in\Gamma_f$ s'expressa com $x=p+\vv{px}$ \\
	Aplicant la $f$ tenim
	\[x=f(x)=f(p+\vv{px})=f(p)+\widetilde{f}(\vv{px})=p+\widetilde{f}(\vv{px})\]
	\[\vv{px}=\widetilde{f}(\vv{px})\implies(\widetilde{f}-1)(\vv{px})=0\implies\vv{px}\in Ker(\widetilde{f}-1)\]
\end{proof}

\begin{exmp}\begin{enumerate}
	\item Translació $\implies\Gamma_f=\varnothing$
	\item Homotècia $\implies\Gamma_f=centre$
	\item Simetria $\implies\Gamma_f=\mathbb{L}=a+F$
	\item Projecció $\implies\Gamma_f=\mathbb{L}=a+F$
\end{enumerate}\end{exmp}

\begin{prop}2.
	Si $\widetilde{f}$ no té el 1 com a valor propi, llavors $\Gamma_f=\{p\}$, o sigui $f$ té un únic punt fix.
\end{prop}
\begin{proof}
	En efecte, tenim un punt $q\in\mathbb{A}$ qualsevol i els possibles punts fixos de $f$,$p$, seràn
	\[p=q+\vv{qp}=f(p)=f(q)+\widetilde{f}(\vv{qp})\implies q-f(q)=\widetilde{f}(\vv{qp})-\vv{qp}=(\widetilde{f}-1)(\vv{qp})\]
	\[\implies \vv{qp}=(\widetilde{f}-1)^{-1}(\vv{f(q)q})\]
	Per tant $\vv{qp}$ és únic
\end{proof}

\begin{defn}[Sistema de punts fixos]
	Sigui $f:\mathbb{A}\to\mathbb{A},\mathcal{R}$ referencia de $\mathbb{A}$, i $M$ la matriu de $f$. Anomenem sistema de punts fixos a $M(X)=X$, que és un sistema de n equacions amb n incognites.
\end{defn}

Probar que si $p_1,\dotsc,p_s$ són punts fixos, tota combinació afí dels $p_i$ també és un punt fix. Sigui $p=\sum\limits_{i=0}^s\lambda_ip_i$, amb $\sum\limits_{i=0}^s\lambda_i=1$
\[f(p)=f(\sum\lambda_ip_i)=\sum\lambda_if(p_i)=\sum\lambda_ip_i=p\]

\begin{defn}
	Diem que $\mathbb{L}$ és una subvarietat lineal invariant per $f$ si $f(\mathbb{L})\subseteq\mathbb{L}$ \\
	Les subvarietats lineals invariants de $dim(\mathbb{L})=0$ són els punts fixos \\
	Les subvarietats lineals invariants de $dim(\mathbb{L})=1$ són els rectes fixes
\end{defn}

\begin{prop}
	Sigui $\mathbb{L}$ una subvarietat lineal de $\mathbb{A}$, llavors $\mathbb{L}$ és invariant per $f\iff(\widetilde{f}(F)\subseteq F)\land(\vv{af(a)}\in F)$
\end{prop}
\begin{proof}
	$\implies )\hspace{2em}$ Suposem $\mathbb{L}$ invariant per $f\implies f(\mathbb{L})\subseteq\mathbb{L}$
	\[f(\mathbb{L})=f(a+F)=f(a)+\widetilde{f}(F)\subseteq a+F\implies\left\{\begin{array}{l}
		\widetilde{f}(F)\subseteq F\\
		\vv{af(a)}\in F\\
	\end{array}\right.\]
	$\impliedby )\hspace{2em}$ Suposem $(\widetilde{f}(F)\subseteq F)\land(\vv{af(a)}\in F)$
	\[f(\mathbb{L})=f(a)+\widetilde{f}(F)\subseteq f(a)+F=a+F=\mathbb{L}\]
\end{proof}

\begin{cor}
	Sigui $\mathbb{L}=a+<u>$ una recta de $A$, llavots $\mathbb{L}$ és invariant per $f\iff(\vv{u}$ és vector propi de $\widetilde{f})\land(\vv{af(a)}\in<u>)\iff(\vv{u}$ és vector propi de $\widetilde{f})\land(rang\left(\begin{array}{c}\vv{af(a)}\\\vv{u}\\\end{array}\right)=1)$
\end{cor}

\begin{exmp}1.
	\[M(f)=\left(\begin{array}{cc|c}
		1&0&2\\
		0&1&1\\ \hline
		0&0&1\\
	\end{array}\right)\]
	Punts fixos?
	\[\left.\begin{array}{r}x+2=x\\y+1=y\\\end{array}\right\}\text{ No té solució}\implies\Gamma_f=\varnothing\] \\
	Rectes fixes? $\vv{u}=(u,v)\neq 0$ és vector propi de $\widetilde{f}$ \\
	Per tant, $\mathbb{L}$ és invariant $\iff rang\left(\begin{array}{c}\vv{af(a)}\\\vv{u}\\\end{array}\right)=1$ \\
	\[f(a,b)=\left(\begin{array}{cc|c}
		1&0&2\\
		0&1&1\\ \hline
		0&0&1\\
	\end{array}\right)\left(\begin{array}{c}a\\b\\1\\\end{array}\right)=\left(\begin{array}{c}a+2\\b+1\\1\\\end{array}\right)\]
	\[\vv{af(a)}=(a+2,b+1)-(a,b)=(2,1)\]
	\[rang\left(\begin{array}{cc}2&1\\u&v\\\end{array}\right)=1\iff\text{Agafem }(u,v)=(2,1)\]
	Per tant les rectes invariants són:
	\[\mathbb{L}=(a,b)+<(2,1)>\text{, amb un }(a,b)\text{ qualsevol}\] 
\end{exmp}
\begin{exmp}2.
	\[M(f)=\left(\begin{array}{cc|c}
		1&1&2\\
		0&1&1\\ \hline
		0&0&1\\
	\end{array}\right)\]
	Punts fixos?
	\[\left.\begin{array}{r}x+y+2=x\\y+1=y\\\end{array}\right\}\text{ No té solució}\implies\Gamma_f=\varnothing\] \\
	Rectes fixes?$\vv{u}=e_1=(1,0)$ és vector propi del valor propi $\lambda=1$ de $\widetilde{f}$ \\
	Per tant, $\mathbb{L}$ és invariant $\iff rang\left(\begin{array}{c}\vv{af(a)}\\\vv{u}\\\end{array}\right)=1$ \\
	\[f(a,b)=\left(\begin{array}{cc|c}
		1&0&2\\
		0&1&1\\ \hline
		0&0&1\\
	\end{array}\right)\left(\begin{array}{c}a\\b\\1\\\end{array}\right)=\left(\begin{array}{c}a+b+2\\b+1\\1\\\end{array}\right)\]
	\[\vv{af(a)}=(a+b+2,b+1)-(a,b)=(b+2,1)\]
	\[rang\left(\begin{array}{cc}b+2&1\\1&0\\\end{array}\right)=2\neq 1\implies\nexists\text{ rectes invariants}\]
\end{exmp}
\begin{exmp}3.
	\[M(f)=\left(\begin{array}{cc|c}
		1&1&0\\
		0&1&0\\ \hline
		0&0&1\\
	\end{array}\right)\]
	Punts fixos?
	\[\left.\begin{array}{r}x+y=x\\y=y\\\end{array}\right\}\implies y=0\text{ recta de putns fixos}\] \\
	Rectes fixes? $\vv{u}=e_1=(1,0)$ és vector propi del valor propi $\lambda=1$ de $\widetilde{f}$ \\
	Per tant, $\mathbb{L}$ és invariant $\iff rang\left(\begin{array}{c}\vv{af(a)}\\\vv{u}\\\end{array}\right)=1$ \\
	\[f(a,b)=\left(\begin{array}{cc|c}
		1&1&0\\
		0&1&0\\ \hline
		0&0&1\\
	\end{array}\right)\left(\begin{array}{c}a\\b\\1\\\end{array}\right)=\left(\begin{array}{c}a+b\\b\\1\\\end{array}\right)\]
	\[\vv{af(a)}=(a+b,b)-(a,b)=(b,0)\]
	\[rang\left(\begin{array}{cc}b&0\\1&0\\\end{array}\right)=1\forall a,b\]
	Per tant les rectes invariants són:
	\[\mathbb{L}=(a,b)+<(2,1)>\text{, amb un }(a,b)\text{ qualsevol}\]
\end{exmp}

\begin{defn}
	Sigui $\mathbb{A}$ un espai afí amb $dim(\mathbb{A})=n$, un hiperpla $\mathbb{H}$ és una subvarietat lineal de $dim(\mathbb{H})=n-1$
\end{defn}

\begin{prop}
	Sigui $f:\mathbb{A}\to\mathbb{A}$ una aplicació afí, que en una referència $\mathcal{R}$ té matriu $M$. Llavors, els hiperplans invariants per $f$ són els de l'equació
	\[A_1x_1+\dotsb+A_nx_n+B=0\]
	amb $(A_1,\dotsc,A_n,B)$ un vector propi de $M^T$ i $(A_1,\dotsc,A_n)\neq 0$
\end{prop}
\begin{proof}
	Anomenem $V=(A_1 \dotsc A_n B)$ i $X=\left(\begin{array}{c}x_1\\\vdots\\x_n\\1\\\end{array}\right)$, de forma que la equació del hiperpla
	\[A_1x_1+\dotsb+A_nx_n+B=0\iff V\cdot X=0\]
	Com $M$ és la matriu de $f$, les coordenades de $f(x)=y$ en funció de les $x$ són, si $X$ són les coordenades de $x$, $Y$ les coordenades de $y$:
	\[Y=M\cdot X\]
	Per tant si $\mathbb{H}$ és l'hiperpla $VX=0$
	\[f^{-1}(\mathbb{H})=\{x\in\mathbb{A}:f(x)\in\mathbb{H}\}=\{X:V\cdot M\cdot X=0\}\]
	Així $\mathbb{H}$ és invarriant $\iff f(\mathbb{H})\subseteq\mathbb{H}\iff\mathbb{H}\subseteq f^{-1}(\mathbb{H})$ \\
	Per tant el sistema
	\[\left.\begin{array}{r}VMX=0\\VX=0\\\end{array}\right\}\text{ és de rang 1 }\iff VM=\lambda V\]
	Com $V$ són vectors fila, transposem tota la equació
	\[M^TV^T=\lambda V^T\]
	On $V^T$ és un vector director propi de $M^T$ amb $(A_1,\dotsc,A_n)\neq 0$
\end{proof}

\begin{exmp}
	Trobar els hiperplans invariants de l'aplicació
	\[M(f)=\left(\begin{array}{ccc|c}3&1&0&0\\0&2&1&1\\0&0&2&0\\\hline 0&0&0&1\\\end{array}\right)\]
	1. Trobem $M^T$
	\[M(f)=\left(\begin{array}{ccc|c}3&0&0&0\\1&2&0&0\\0&1&2&0\\\hline 0&1&0&1\\\end{array}\right)\]
	2. Trobem els valors propis de $M^T$: 3, 2 i 1\\
	veps de 1: \\
	\[e_4=(0,0,0,1)\implies\nexists\text{ hiperplans invariants per }\lambda=1\]
	veps de 3: \\
	\[Ker(M^T-3I)=Ker\left(\begin{array}{ccc|c}0&0&0&0\\1&-1&0&0\\0&1&-1&0\\\hline 0&1&0&-2\\\end{array}\right)\implies\left.\begin{array}{r}x-y=0\\y-z=0\\y-2t=0\\\end{array}\right\}\implies V_3=(1,1,1,\frac{1}{2})\]
	\[\implies x+y+z+\frac{1}{2}=0\]
	veps de 2: \\
	\[\cdots\]
\end{exmp}

\begin{prop}
	$\forall f:\mathbb{A}^n\to\mathbb{A}^n$ aplicació afí, $\exists$ una referencia (de Jordan) tal que la matriu de $f$ en aquesta referència és una matriu de Jordan superior
\end{prop}

Aplicacions afins de dim 1

Sigui $f:\mathbb{A}\to\mathbb{A}$ una aplicació afí en una referència de Jordan $\mathcal{R}$, amb $dim(A)=1$, la matriu $M$ podria ser dels següents tipus:
\begin{itemize}
	\item Si $f$ té punts fixos és
	\begin{itemize}
		\item La identitat $\iff\left(\begin{array}{c|c}1&0\\\hline0&1\\\end{array}\right)$ 
		\item Una homotècia de rao $\lambda\neq 1\iff\left(\begin{array}{c|c}\lambda&0\\\hline0&1\\\end{array}\right)$
	\end{itemize}
	\item Si $f$ no té punts fixos és una translació $\iff\left(\begin{array}{c|c}1&1\\\hline0&1\\\end{array}\right)$
\end{itemize}

Aplicacions afins de dim 2

Sigui $f:\mathbb{A}\to\mathbb{A}$ una aplicació afí en una referència de Jordan $\mathcal{R}$, amb $dim(A)=2$, la matriu $M$ podria ser dels següents tipus:
\begin{itemize}
	\item $\left(\begin{array}{cc|c}1&&\\0&1&\\\hline0&0&1\\\end{array}\right)$
	\begin{itemize}
		\item $\left(\begin{array}{cc|c}1&0&\\0&1&\\\hline0&0&1\\\end{array}\right)\left\{\begin{array}{l}
			\text{Amb punts fixos: }\\
			\left(\begin{array}{cc|c}1&0&0\\0&1&0\\\hline0&0&1\\\end{array}\right)\iff f\text{ identitat}\\
			\text{Sense punts fixos: }\\
			\left(\begin{array}{cc|c}1&0&0\\0&1&1\\\hline0&0&1\\\end{array}\right)\iff f\text{ translació}\\
		\end{array}\right.$
		\item $\left(\begin{array}{cc|c}1&1&\\0&1&\\\hline0&0&1\\\end{array}\right)\left\{\begin{array}{l}
			\text{Amb punts fixos: }\\
			\left(\begin{array}{cc|c}1&1&0\\0&1&0\\\hline0&0&1\\\end{array}\right)\iff f\text{ homologia especial}\\
			\text{Sense punts fixos: }\\
			\left(\begin{array}{cc|c}1&1&0\\0&1&1\\\hline0&0&1\\\end{array}\right)\iff f\text{ composició de una homologia especial i una translació}\\
		\end{array}\right.$
	\end{itemize}
	\item $\left(\begin{array}{cc|c}\lambda&0&\\0&1&\\\hline0&0&1\\\end{array}\right)\left\{\begin{array}{l}
		\text{Amb punts fixos: }\\
		\left(\begin{array}{cc|c}\lambda&0&0\\0&1&0\\\hline0&0&1\\\end{array}\right)\iff f\text{ homologia general}\\
		\text{Sense punts fixos: }\\
		\left(\begin{array}{cc|c}\lambda&0&0\\0&1&1\\\hline0&0&1\\\end{array}\right)\iff f\text{ composició d'homologia general i translació}\\
	\end{array}\right.$
	\item $\left(\begin{array}{cc|c}\lambda&&\\0&\mu&\\\hline0&0&1\\\end{array}\right)$
	\begin{itemize}
		\item Amb punts fixos: $\left(\begin{array}{cc|c}\lambda&0&0\\0&\lambda &0\\\hline0&0&1\\\end{array}\right)\iff f$ homotècia
		\item Amb punts fixos: $\left(\begin{array}{cc|c}\lambda&1&0\\0&\lambda &0\\\hline0&0&1\\\end{array}\right)$
		\item Amb punts fixos: $\left(\begin{array}{cc|c}\lambda&0&0\\0&\mu &0\\\hline0&0&1\\\end{array}\right)$
	\end{itemize}
\end{itemize}

\section{Espais euclidis}

\subsection{Producte escalar}

Sigui $E$ un e.v. sobre $\mathbb{R}$

\begin{defn}
	Un producte escalar sobre $E$ és una aplicació $\phi:E\times E\to\mathbb{R}$ tal que
	\begin{itemize}
		\item $\phi$ bilineal $\iff\left\{\begin{array}{rcl}\phi(\vv{x}+\vv{x}',\vv{y})&=&\phi(\vv{x},\vv{y})+\phi(\vv{x}',\vv{y})\\\phi(\lambda \vv{x},\vv{y})&=&\lambda\phi(\vv{x},\vv{y})\\\phi(\vv{x},\vv{y}+\vv{y}')&=&\phi(\vv{x},\vv{y})+\phi(\vv{x},\vv{y}')\\\phi(\vv{x},\lambda \vv{y})&=&\lambda\phi(\vv{x},\vv{y})\\\end{array}\right.$
		\item $\phi$ simètrica $\iff\phi(\vv{x},\vv{y})=\phi(\vv{y},\vv{x})$
		\item $\phi$ definida positiva $\iff(\vv{x}=\vv{0}\iff\phi(\vv{x},\vv{x})=0)\land(\vv{x}\neq \vv{0}\implies\phi(\vv{x},\vv{x})>0)$
	\end{itemize}
	Denotem el producte escalar per $\phi(\vv{x},\vv{y})=\vv{x}\cdot\vv{y}=<\vv{x},\vv{y}>=<\vv{x}|\vv{y}>$
\end{defn}


